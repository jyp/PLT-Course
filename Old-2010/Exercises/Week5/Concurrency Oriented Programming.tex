\documentclass{article}
\usepackage{fancyvrb,hyperref,listings,verbatim}

\newcommand{\Cpp}{C\kern-0.05em\texttt{+\kern-0.03em+}}
\newcommand{\jp}[1]{ \footnote{#1}}

\lstset{commentstyle=\textit,flexiblecolumns=true,mathescape=true}
\lstnewenvironment{erlang}
    {\lstset{language=erlang,frame=tb}}
    {}

\newcommand{\courseinfo}{ John Hughes \\ Programming Paradigms 2008 }
\newcommand{\percent}[1]{\protect \marginpar[l]{\bf [#1 points]}}
\newcommand{\mycomment}[1]{}

\newcounter{question}
\newenvironment{question}[1]{%
  \addtocounter{question}{1}%
  \paragraph{Exercise~\arabic{question}  \percent{#1}}%
}{%
  \vskip1em%
}

\let\longanswer=\comment
\let\endlonganswer=\endcomment
\newcommand{\answer}[1]{}

% Uncomment to show answers
\renewenvironment{longanswer}{\paragraph{Answer~\arabic{question}}}{}
\renewcommand{\answer}[1]{\begin{longanswer}#1\end{longanswer}}
  

\begin{document}
\title{Programming Paradigms 
\\
Concurrency Oriented Programming—Exercises}
\author{\courseinfo}
\date{}
\maketitle

\section{Running Erlang}

You can start an Erlang interpreter using the command \verb!erl!. You
can also install Erlang on your own computer: downloads are available
from erlang.org. Whether you use the version installed at Chalmers, or
install your own, you will need the Erlang documentation. This is
available at \url{http://erlang.org/doc/} (and is also installed along
with the Erlang system when you download a copy to your own computer).


\section{Building a logger}

When an Erlang process crashes, it may be hard to know what the
process was doing before the crash. In this exercise, we will build a
simple logger that enables processes to log events as they run, and
have the last logged event automatically printed if the process later
crashes.

The API you should implement consists of three functions:

\begin{description}
\item[\textsf{logger:start()}]  starts the logger process;
\item[\textsf{logger:stop()}]   stops it;
\item[\textsf{logger:log(Event)}] sends the \textsf{Event} to the logger, and
  informs it that the sending process should be monitored.
\end{description}

For example, the following program uses the logger to report the exit reasons of two processes:
\begin{erlang}
test() ->
    spawn(fun() ->
	  logger:log(starting_1),
	  Pid = spawn_link(fun() ->
			   logger:log(starting_2),
			   timer:sleep(1000),
			   logger:log(stopping_2),
			   1/0
		   end),
	  logger:log(started_2),
	  timer:sleep(100),
	  logger:log(about_to_kill_2),
	  exit(Pid,reason),
	  timer:sleep(100),
	  logger:log(stopping_1),
	  exit(done)
	end).
\end{erlang}
The output when this program is run should be something  like this:
\begin{verbatim}
25> logger:test().
<0.100.0>
Exited: <0.101.0> with reason reason
Last message: starting_2
Exited: <0.100.0> with reason reason
^^^_
Last message: about_to_kill_2
\end{verbatim}

(Try removing the calls to sleep, and see the effect on the output).

\paragraph{Hints}
First, note that crossing module boundaries does not mean crossing process boundaries.
Also, note that you can use ``self'' and ``link'' anywhere in the functions of the logger module.
The functions ``register'' and ``where\_is'' are also useful.


\begin{question}{25}
    Write a logger module with the API defined above. 
    Pay attention to the following:
    \begin{itemize}
        \item Your logger process must be able to track multiple (client) processes.
        \item Do not use the functions \textsf{put} and \textsf{get}.
        \item Try not to use \textsf{on\_exit} directly.
        \item There should be at most one logger process running at a time.
    \end{itemize}
\end{question}

\section{Fault-tolerance by Replication}

Sooner or later, every computer crashes. Fault-tolerant systems should
continue to work \emph{even though} the computers they run on crash. The only
way to achieve this is for critical servers to be replicated, running
on \emph{more than one} computer. In this exercise, you will develop a simple
replicated server, although, for simplicity, we will run both copies
of the server in the same Erlang node.

The intention is that you should develop an enhanced version of the
simple generic server I developed in the lecture, which automatically
starts \emph{two} servers rather than one. One of them will be the \emph{main}
server, registered under the specified name, and this will take care
of the requests that clients make. The other will be a \emph{backup} server,
which also processes the requests, but does not send replies to the
clients. If the main server should crash, then the backup server steps
in and takes over, becoming the main server and spawning a new backup
server. Of course, the main server should also monitor the backup
server, and restart it too if necessary.

The effect should be that either server may crash without any effect
on the behaviour that clients see (provided no requests are made
during the short interval between the main server crashing, and the
backup server taking over its work).

You should achieve this \emph{only} by changes in the generic server
module---the callback modules ought to be entirely unaffected by this
extension. You can start from the module in \textsf{server1.erl} in
the course handouts.

\begin{question}{35}
    Write a fault-tolerant server module as described above. 
    Pay attention to the following:
    \begin{itemize}
        \item The backup process should be registered under a different name.
        \item Make sure that the backup server takes over the role of Main
              (instead of spawing a new Main).
        \item The backup server must not send replies!
    \end{itemize}
\end{question}

\section{Transactional Servers}

In the lecture, we saw how to build a generic server with
\emph{transactional semantics}, in that crashing requests simply leave the
server state unchanged. More generally, we might want to perform a
\emph{sequence} of requests to the server in one ``transaction''. In general, a
transaction is a sequence of requests which \emph{appear to be atomic} to
other processes. Thus, if we start a transaction, perform a sequence
of requests, and then end the transaction, then other clients of the
same server should only see server states before the transaction
starts, or after it ends---no intermediate state should ever be visible
to another client.

One way to achieve this would be to ``lock the server'', and simply
queue up requests from other clients until the entire transaction is
over.  But since a transaction may run for quite a long time, then
this could restrict concurrency dramatically. Instead, transactions
are usually implemented using \emph{optimistic concurrency}---which mean that
two or more are allowed to proceed in parallel, but at the end of a
transaction, a check is made to see whether or not any interference
actually occurred, and if so, then the transaction is \emph{aborted}.

While a transaction is running, then its effects must not be visible
to other clients, which means that the server must copy the server
state when the transaction starts, and then perform the requests that
make up the transaction on this copy, instead of on the main server
state. In this way, other clients will see no state changes while the
transaction is running.


When a transaction ends, then the copied state cannot simply replace
the main state, because other operations may have changed the main
state in the meantime. Instead, we must \emph{apply the requests in the
transaction} to the main state at this point—which we do atomically,
thus guaranteeing that other clients see the state either before or
after the entire transaction.  Thus the server needs to record the
requests that make up a transaction, so that they can be replayed when
the transaction ends.

There is a problem, though. If another client has changed the server
state while a transaction was running, then it's possible that
replaying the requests will generate \emph{different} replies from those that
have already been sent to the transaction client. If this happens,
then the transaction cannot be completed, because the transaction
client has already seen replies that differ from those that would be
generated by performing the transaction atomically—so there is no way
to make the transaction as a whole appear to be atomic. Instead, the
transaction is aborted, and the client is informed. Usually, clients
perform transactions in a loop, so that if the transaction aborts,
then it is just repeated---perhaps after a delay.

Your task is to modify the generic server in \textsf{server1.erl} to support
two new requests: \textsf{start\_transaction}, and
\textsf{end\_transaction}. The first, \textsf{start\_transaction},
indicates that the calling client is starting a transaction, and
subsequent requests from the same client pid should be processed by
the server as part of this transaction. When the client eventually
sends an \textsf{end\_transaction} request, then the transaction
should be applied atomically to the main server state, provided there
are no conflicts, and a Boolean returned to indicate whether or not
the transaction succeeded.

Once again, your changes should be made \emph{only} to the generic server
code---the callback modules should be entirely unaffected. In this way,
transactions can be added to any server at all, without changing the
callback code in the slightest.

\begin{question}{40}
    Write a transactional server module as described above. 
    Pay attention to the following:
    \begin{itemize}
        \item Do not split the state into individual variables.

\item The server should reply immediately to requests (not wait until the end
of the transaction)

        \item The transaction should only fail if different replies are generated,
          not just because the state has changed.
    \end{itemize}

\end{question}

\paragraph{Note}

A very useful extension is to allow transactions to span over \emph{multiple}
servers---so that a client can, for example, withdraw money from one
bank account, and deposit it into another, all as part of the same
transaction, to guarantee that either both effects happen, or neither
does. This can be implemented via a \emph{two-phase commit}, where the client
first asks each server whether it can commit the transaction, then if
all reply positively, instructs each server to actually do so. If any
server cannot complete the transaction, then the client instructs all
of them to abort it. Why not implement this too, if you have time to
spare?

\end{document}
