\documentclass{article}

\input{header}


\begin{document}
\title{Programming Paradigms 
\\
Ex. 3: Imperative Programming}
\author{\courseinfo}
\date{}

\maketitle
\mycomment{
\paragraph{Submission}
XXX
We will use the  Fire lab report system (\textsf{http://fire.cs.chalmers.se}) 
for
handling submissions and grades. Before you can submit your answers,
you need to register in the Fire system: team up with one more student and 
register your  group (see ``Documentation for students'' on
the webpage of Fire). Note that the system is not yet accessible for
our course, but is expected to be so later next week.


Submit your answers (one per pair of students)
electronically before the deadline.
%
\textit{To pass the course, you must pass 25\% of all problems on \textbf{this}
sheet and 50\% of all problems total. For some problems, you may earn
extra points.}

\begin{center}
\fbox{Deadline: Thursday, Nov. 8, 2007 XXX, midnight (Swedish time).}
\end{center}
%
The deadline is not extensible. 

\paragraph{Academic integrity} is important. If you submit solutions that
are not your own, we hand the case over to the Chalmers officials, as specified
here 
\begin{center}
\textsf{http://www.chalmers.se/en/sections/education/current\_students/}\\
\textsf{\-joint\_rules\_and\_dire/rules\_of\_discipline} (English) 
%
and \\
%
\textsf{http://www.chalmers.se/sections/ar\_student/regler\_och\_riktlinje/disciplinstadga}(Swedish).
\end{center}

\paragraph{Help} Don't hesitate to ask for help!
The course assistants are 
Gustav Munkby and Jean-Philippe Bernardy.
You can reach them at the following email addresses:
%
\begin{center} 
\textsf{gustav.munkby-at-chalmers.se}  and 
\textsf{bernardy-at-chalmers.se} 
\end{center} 
%
(replace ``-'' by  ``.'', and ``at'' by the at-sign).


\newpage
}

\section{Hello, Fortress [25p]}
Among the recent languages is the language 
 \textsf{Fortress}, developed by Sun within
the \textsf{High Productivity Computer Systems (HPCS)} competition
by the US goverment.
It is a general-purpose language but has features---built-in support
for parallelism, for example---that make it interesting for 
 scientific computing and high-performance computations.
Although \textsf{Fortress} has been around for a few years, its 
specification (v.1.0) has been published only this year.
The purpose of this exercise is to expose you to language
specifications (and the challenges of language development).
As example, we use  a 
method from one of the
Fortress tutorials for computing the conjugate gradient: 
%


\begin{java}
//Fortress Programming Language Tutorial, PLDI, 11 June 2006 31 
// � 2006 Sun Microsystems, Inc. All rights reserved. 
// Simple Example: NAS CG Kernel (ASCII) 
conjGrad[\Elt extends Number, nat N,Mat extends Matrix[\Elt,N BY N\],
          Vec extends Vector[\Elt,N\]\](A: Mat, x: Vec): (Vec, Elt) = do
   cgit_max = 25
   z: Vec = 0
   r: Vec = x
   p: Vec = r
   rho: Elt = r^T r
   for j <- seq(1:cgit_max) do
     q = A p
     alpha = rho / p^T q
     z := z + alpha p
     r := r - alpha q
     rho0 = rho
     rho := r^T r
     beta = rho / rho0
     p := r + beta p
   end
  (z, ||x - A z||)
end

(z,norm) = conjGrad(A,x) 
\end{java}

Download the current language specification and answer the
following questions:
\begin{enumerate}
\item What is the difference between the \textit{reach} and the
\textit{scope} of a declaration? Look up the language specification, then give
a definition in your own words. Use the \textsf{conjGrad}
example above to illustrate the
two terms.

\item Explain the difference of the following five variable
declarations in terms of mutability and initialization and, again,
use the \textsf{conjGrad} example above for illustration:
\[
\begin{array}{l}
x = e \\
x: T = e  \\
\mbox{var\ } x: T = e\\
\mbox{var?\ } x: T := e\\
(x,y) = (e,f)
\end{array}
\]

\mycomment{
\[
\begin{array}{l}
\pi = 3.14 \\
\rho: \mbox{Float} = r^\pi  \\
\mbox{var\ } x: \mbox{Double} = 5.555 \\
\mbox{var?\ } x: \mbox{Double} := 5.555\\
(x,y) = (1,2)
\end{array}
\]
}
\item Consider the signature of the method \textsf{conjGrad}, which
consists of 3 parameter lists, namely [...], (..), and (..). Which part
refers to program parameters that can be passed at run time?
Which parameter-passing mechanism(s) does \textsf{Fortress}
support, and if there is more than one,  which one applies here?
Back your answer with a reference to the language specification
so that we know that you are not just guessing. 

\item For-loops in Fortress are specified as follows (13.5):
\begin{quote}
A \textsf{for} loop consists of \textsf{for} followed by a generator 
clause list (discussed in Section 13.14), followed by a non-parallel 
do expression (the loop body; see Section 13.11). 
Parallelism in \textsf{for} loops is specified by the generators 
used (see Section 13.14); in general the programmer must assume that 
each loop iteration will occur independently 
in parallel unless every generator is explicitly sequential. 
For each iteration, the body expression is evaluated in the 
scope of the values bound by the generators. The value and type of a 
\textsf{for} loop is () . 

\end{quote}
Explain how the \textsf{for}-loop in the \textsf{conjGrad} method
works: what is the ``generator clause list''; what is the generator;
what does it mean that ``the body expression is evaluated in the scope of
the values bound by the generators''; and in which
order are values generated? 



%http://projectfortress.sun.com/Projects/Community/wiki/FortressLanguageSpecificationErrata
\end{enumerate}

Hint: for most answers, you should consult part II of the language specification (``Fortress for application programmers'').

\paragraph{Further reading}
\begin{itemize}
\item The main project page: \\
\url{http://projectfortress.sun.com/Projects/Community}
\item The language specification:
\url{
http://projectfortress.sun.com/Projects/Community/browser/trunk/Specification/fortress.1.0.pdf}
\end{itemize}

\newpage
\section{Parameter passing [25p]}
\begin{enumerate}
\item For each mechanism of parameter passing, determine the final
values of the array \textsf{A} after the call to \textsf{f}. 

\begin{java}
void f(int x, int y) {
    x = 1;
    y = 3;
}

int[] A = new int[2];
A[0] = 0;
A[1] = 2;
f(A[0],A[A[0]]);
\end{java}
\item An important question of call-by-result, which was not discussed in
class, is the question \textit{when} the lvalue of the actual parameter is 
computed. One might expect that this happens before the callee executes,
but that is not necessarily the case---since the value of the actual
parameter is not needed until the 
callee returns, its computation could be delayed until the value is needed.
Ada, for example,  supports early lvalue-computation, while Algol-W implements
late lvalue-computation.

Use the snippet below to explain the difference between early and late
lvalue-computation.
\begin{java}
void f(int [] A) {
   int i = 0;
   g(i, A[i]);
}
void g(by-reference int i, by-result int j) {
   j = 2;
   i = 1;
   j = 3;
}
\end{java}

% When does C\# perform the lvalue-computation for its by-result parameters?

\item Another important detail of call-by-result and call-by-value-result concerns
the order in which the final values of the formal parameter are written back
to the actual parameters. For a binary method \textsf{f(x,y)}, for
example, either (the value of) \textsf{x} or (the value of) 
\textsf{y} is written back
first; this order might make a difference. 

Write two methods \textsf{f,g} (in pseudo-code,
if you wish) that demonstrate the difference.
\end{enumerate}

\newpage
\section{Activation records and scoping [25p]}
The \textit{scope} of a variable declaration refers to the range of
statements in which the variable can be referenced. 
\begin{itemize}
\item In \textit{static scoping}, the use of a variable $y$ in
a subprogram $f$ refers to the declaration of $y$ at the beginning of $f$ 
or, if not
declared there, to the declaration of $y$ in the subprogram $g$ 
whose declaration
contains the declaration of $f$, and so on. The static scope, thus,
is a property of the program text and can be determined at compile time.
\item In \textit{dynamic scoping}, on the other hand, 
the scope depends on the call sequence of
subroutines: the use of a variable refers to the declaration in 
the nearest caller (``dynamic parent'').
\end{itemize}
Java, C, \Cpp, C\#, and most other
 languages support static scoping. The language Perl is special
since it gives the user control over  scoping: 
variables declared as \textsf{`my'} are statically scoped, 
variables declared as \textsf{'local'} are dynamically scoped.  

\mycomment{
In this exercise, the question is: given a particular  use of
a variable, how does the run-time system know which declaration that 
variable refers to? As discussed in class, activation records contain
only two kinds of internal information: the return address and the previous
activation record. This does not suffice for languages with
static scoping and \textit{non-local variable references}.
}

\begin{enumerate}
\item What is the effect of static and dynamic scoping?
Consider the Perl program below and vary  between 
\textsf{my} and \textsf{local}
for the three
declarations of $x$. When, if at all, does the value $x=2$ get printed?
When $x=4.14$ or $x=56$ (if at all)? 
(Hint: Perl is installed on (almost) all machines; you can run
the program.)
\item In dynamic scoping, it is easy for the run-time system
to determine which declaration
the use of a variable refers to. Explain. (Hint: run-time
stack)
\item \textbf{[3 Bonuspoints]} Static scoping with non-local references, on the other hand,
is not so easy to implement. In particular, the kind of 
activation records discussed in class (which contain
only two kinds of internal information: the return address and the previous
activation record) does not suffice. 
%

What information would be necessary and where could it be stored? 
Make a suggestion  and 
sketch a (simple) algorithm that takes a run-time stack and an activation record,
and returns the ``right'' reference (or at least its activation record).
How expensive is your algorithm? Determine the parameters on which its
cost depends and give the cost of the algorithm.
\end{enumerate}


\begin{perl}
 sub outer {
   my $x = 0;

   sub r(){
       $x = $x + 1; 
       print("In subroutine r, x = ", $x);
   }
   sub q() {
      my $x = 3.14; 
      &r;      # call outer's subroutine r
   }
   sub p() {
      my $x = 55; 
      &q;      # call outer's subroutine q
   } 
   &p;         # call subroutine p
}

&outer;
\end{perl}

\newpage
\section{Duff's device [25p]}
Duff's device is an optimization idiom for serial copies 
in the language C. The fragment below lists first the original code,
then Duff's optimization.
\begin{cplus}
/* (Almost) original code */
int main() {
   register short *to, *from;
   register int count;	
   ... 
   {
     /* pre: count > 0 */		
     do
	  *to++ = *from++;		
     while(--count>0);	
   }
   return 0; 
}

\end{cplus}

\begin{cplus}
/* Duff's transformation */
int main() {
  register short *to, *from;
  register int count;	
  ...
  {
     /* pre: count > 0 */		
     int n = (count + 7) / 8;
     switch(count % 8){
       case 0:	do{	*to++ = *from++;
       case 7:	        *to++ = *from++;
       case 6:	        *to++ = *from++;
       case 5:	        *to++ = *from++;
       case 4:	        *to++ = *from++;
       case 3:	        *to++ = *from++;
       case 2:	        *to++ = *from++;
       case 1:	        *to++ = *from++;
                } while (--n > 0);
     }
  }
  return 0;
}
\end{cplus}
Your task
\begin{enumerate}
%\item Explain pointer variables in terms of r-values and l-values: what is the r-value of a pointer?
\item What happens in the assignment ``\textsf{*to++ = *from++;}''?  Explain. 
\item Look up the storage class \textsf{register}. Furthermore, 
look up what the   \textsf{switch} statement does (preferably
in the language specification) and briefly
explain the second program.
\item Why did Duff expect the second program to be faster? Speculate and
give a brief answer.
\end{enumerate}

\paragraph{Further reading}
\begin{itemize}
\item For the original email in which Tom Duff presented his ``device'',
see 

\textsf{http://www.lysator.liu.se/c/duffs-device.html}
\item The specification of the C language is available at 

\textsf{http://www.open-std.org/jtc1/sc22/wg14/www/docs/n1124.pdf}
\item Check the generated code yourself: compile it
with \textsf{gcc -S $<$filename$>$}.
\end{itemize}
\end{document}
\section{Answers (sketches)}



\section*{Fortress}
\begin{enumerate}
\item Reach vs. scope: language spec, section 7.3
\item Variable declarations: 8.2
\item Parameter passing: call-by-value, 13.6
\item For-loops: 13.15
\end{enumerate}

\section*{Parameter passing}
\begin{enumerate}
\item Elements in A

\begin{tabular}{lll}
               & A[0] & A[1] \\
 by-reference  & 3    & 2    \\
by-value        & 0   & 2   \\
by-result      &  3 (1) & 2 (3);  depends on order, see next problem \\
by-value-result & 1 & 3 \\
by-name      & 1 & 3  \\
\end{tabular}
\item Early/late:

\begin{tabular}{lll}
               & A[0] & A[1] \\
early          & 3    & ? \\
late       & ?   & 3   \\
middle     &    & \\
\end{tabular}
\item Order:
\begin{java}
int f(int x, int y) {
    x = 4;
    y = 5;
}
int a=3;
p(a,a); 
\end{java}

\end{enumerate}

\section*{Scoping}
\begin{enumerate}
\item Scoping declaration in outer/p/q
\begin{itemize}
\item my/my/my  or local/my/my -$>$ x=1
\item local/local/local or local/my/local -$>$ x =4.14
\item local/local/my -$>$ x=55
\end{itemize}
\item Traverse the run-time stack, take the first declaration encountered.
\item (Not-constant solution:) Include in the activation record a 
``static link'' (a.k.a. access links), then follow the link to the
static parent. For a solution in constant time: compute the
``static distance'' between a variable definition and a variable
reference; jump to the definition in constant time.
\end{enumerate}

\section*{Duff's device}
\begin{enumerate}
\item A pointer is variable whose r-value is a l-value
\item ``register'' is implementation-dependent; the switch statement
is infamous because only an explicit ``break'' returns from the 
switch. Without ``break'', execution continues through all remaining cases.
\item Suspected optimization: loop unrolling 
\end{enumerate}

\end{document}
