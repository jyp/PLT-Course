\documentclass[7pt]{article}
\usepackage{graphicx}
\usepackage{amsmath}
\usepackage[mathletters]{ucs}
\usepackage[utf8x]{inputenc}
\usepackage{listings}
\usepackage{geometry}

\lstnewenvironment{code}{\lstset{language=Haskell,basicstyle=\small\ttfamily}}{}

\setlength{\parindent}{0pt}
\setlength{\parskip}{6pt plus 2pt minus 1pt}

\usepackage{pgf, tikz}
\usetikzlibrary{shapes}
\usetikzlibrary{shapes.multipart}
\usetikzlibrary{arrows}

\lstdefinelanguage{FSharp}
      {morekeywords={let, new, match, inherit, abstract, with, rec,
          open, module, namespace, type, of, member, and, for, in, do,
          begin, end, fun, function, try, mutable, if, then, else,
          class, interface, end},
    keywordstyle=\color{blue},
    basicstyle = \small,
    sensitive=false,
    morecomment=[l][\color{green!50!black!80}]{///},
    morecomment=[l][\color{green!50!black!80}]{//},
    morecomment=[s][\color{green!50!black!80}]{{(*}{*)}},
    morestring=[b]",
    stringstyle=\color{red},
    columns=fullflexible
    }


\lstdefinelanguage{Smalltalk}{
  basicstyle=\ttfamily,
  keywordstyle=\bfseries,
  morekeywords={self,super,true,false,nil,thisContext}, % This is overkill
  morestring=[d]',
  morecomment=[s]{"}{"},
  alsoletter={\#:},
  escapechar={!},
}[keywords,comments,strings]


\newcommand{\Blue}[1]{\color{blue}#1\color{black}\xspace}


\usepackage{array}
% This is needed because raggedright in table elements redefines \\:
\newcommand{\PreserveBackslash}[1]{\let\temp=\\#1\let\\=\temp}
\let\PBS=\PreserveBackslash

%%%%%%%%%%%%%% 
\newcommand{\solution}[1] {}
%\newcommand{\solution}[1] {\textbf{Solutions:}\\ #1}

%\newcommand{\comment}[1]{}
\newcommand{\comment}[1]{\marginpar{#1}}


\begin{document}
\newcommand{\examtime}{14:00, March 9th, 2012}
\newcommand{\points}[1]{\marginpar{\bf #1 points}}


\section{Explicit stacks}

Consider the function
\begin{verbatim}
procedure fib(n)
  if n == 0
    return 0
  else if n == 1
    return 1
  else 
    return fib (n-1) + fib (n-2)
\end{verbatim}

Remove explicit recursion, using an explicit stack. 

Remarks:

\begin{itemize}
\item  You should assume a ``global'' stack accessed via \texttt{push}
  and \texttt{pop} primitives.

\item You must implement \emph{the same algorithm}. Do not change
  the algorithm in the process of doing the translation.

\item If tail-call optimisation applies to this example, your
  translation must use it.
\end{itemize}

Hints: you should first define the type of values that you can push on
the stack, and you can assume that labels can be used as values.

\section{Continuations}

Consider the following function, which computes the factorial of its argument.
\begin{verbatim}
fac 0 = return 1
fac n = do r <- fac (n-1)
           yield
           return (r*n)
\end{verbatim}

The above algorithm will be used on very big numbers, hence
potentially running for a long time, concurrently with other
processes. Therefore, the implementer of \texttt{fac} has decided
to insert a call to \texttt{yield}, which performs no useful task, but
gives the opportunity for the runtime environment to schedule another
processes. (The computation of the factorial will be continued
later.)

Transform \texttt{fac} to use explicit continuations.

Remarks: 
\begin{itemize}
\item In the translation, you will use a different version of
  \texttt{yield}, which takes an explicit continuation as argument.
\item The translation of \texttt{fac} should also take an explicit
  continuation as argument.
\end{itemize}

\section{Laziness}
In Haskell you can write a matrix comprehension like so:
\begin{verbatim}
matrix = array bounds [((x,y),valueAtIndex x y) | (x,y) <- range bounds]
  where bounds =  ((low_x,low_y),(high_x,high_y))
\end{verbatim}

You can then access a cell $x,y$ in the array using \texttt{matrix!(x,y)}.
%
For example, in this matrix of $100$ elements the value at index $(x,y)$ contains $x ×
y$: \texttt{example!(x,y) == x * y}.
\begin{verbatim}
example = array bounds [((x,y),x * y) | (x,y) <- range bounds]
  where bounds = ((1,1),(10,10))
\end{verbatim}

Consider the following algorithm to compute the Levenshtein distance between two strings:
{\small
\begin{verbatim}
int LevenshteinDistance(char s[1..m], char t[1..n])
{
  // for all i and j, d[i,j] will hold the Levenshtein distance between
  // the first i characters of s and the first j characters of t;
  // note that d has (m+1)x(n+1) values
  declare int d[0..m, 0..n]

  for i from 0 to m
    d[i, 0] := i // the distance of any first string to an empty second string
  for j from 0 to n
    d[0, j] := j // the distance of any second string to an empty first string
  for j from 1 to n {
    for i from 1 to m {
      if s[i] = t[j] then
        d[i, j] := d[i-1, j-1]       // no operation required
      else
        d[i, j] := minimum (
                     d[i-1, j] + 1,  // a deletion
                     d[i, j-1] + 1,  // an insertion
                     d[i-1, j-1] + 1 // a substitution
                   )
    }
  }
  return d[m,n]
}
\end{verbatim}
}

Using \emph{a single comprehension}, construct a matrix \texttt{d},
such that \texttt{d!(i,j)} is the Levenshtein distance between the
first \texttt{i} characters of \texttt{s} and the first \texttt{j}
characters of \texttt{t}, assuming
\texttt{s :: String} and \texttt{t :: String}.

\begin{itemize}
\item The distance between any two substrings must not be computed
  twice.
\item You may also assume a predefined function \texttt{minimum :: [Int] -> Int}
\item You can access the $\mbox{\texttt{i}}^{\mbox{th}}$ character of
  string \texttt{s} with the expression \texttt{s!!(i-1)}.
\end{itemize}

\end{document}
