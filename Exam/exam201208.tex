\documentclass{article}
\usepackage{graphicx}
\usepackage{amsmath}
\usepackage[utf8]{inputenc}
\usepackage{listings}
\lstnewenvironment{code}{\lstset{language=Haskell,basicstyle=\small\ttfamily}}{}

\setlength{\parindent}{0pt}
\setlength{\parskip}{6pt plus 2pt minus 1pt}

\usepackage{pgf, tikz}
\usetikzlibrary{shapes}
\usetikzlibrary{shapes.multipart}
\usetikzlibrary{arrows}

\lstdefinelanguage{FSharp}
      {morekeywords={let, new, match, inherit, abstract, with, rec,
          open, module, namespace, type, of, member, and, for, in, do,
          begin, end, fun, function, try, mutable, if, then, else,
          class, interface, end},
    keywordstyle=\color{blue},
    basicstyle = \small,
    sensitive=false,
    morecomment=[l][\color{green!50!black!80}]{///},
    morecomment=[l][\color{green!50!black!80}]{//},
    morecomment=[s][\color{green!50!black!80}]{{(*}{*)}},
    morestring=[b]",
    stringstyle=\color{red},
    columns=fullflexible
    }


\lstdefinelanguage{Smalltalk}{
  basicstyle=\ttfamily,
  keywordstyle=\bfseries,
  morekeywords={self,super,true,false,nil,thisContext}, % This is overkill
  morestring=[d]',
  morecomment=[s]{"}{"},
  alsoletter={\#:},
  escapechar={!},
}[keywords,comments,strings]


\newcommand{\Blue}[1]{\color{blue}#1\color{black}\xspace}


\usepackage{array}
% This is needed because raggedright in table elements redefines \\:
\newcommand{\PreserveBackslash}[1]{\let\temp=\\#1\let\\=\temp}
\let\PBS=\PreserveBackslash

%%%%%%%%%%%%%% 
\newcommand{\solution}[1] {}
%\newcommand{\solution}[1] {\textbf{Solutions:}\\ #1}

%\newcommand{\comment}[1]{}
\newcommand{\comment}[1]{\marginpar{#1}}


\begin{document}
\newcommand{\examtime}{14:00, March 9th, 2012}
\newcommand{\points}[1]{\marginpar{\bf #1 points}}
\noindent
\begin{tabular}{lr}
CHALMERS TEKNISKA H\"OGSKOLA &\examtime{}.\\
Dept. of Computer Science and Engineering & Programming Paradigms\\
Jean-Philippe Bernardy                 & DAT121 / DIT331(GU) \\
\end{tabular}

\vspace{2.5cm} \noindent
\begin{center} {\LARGE
Exam in Programming Paradigms}
\end{center}

\vspace{1.5cm}

\noindent
\examtime{}.\\
\begin{tabular}{ll}
\textbf{Examiner:} &  John Hughes  \\
\textbf{Lecturer:} & Jean-Philippe Bernardy
\end{tabular}
\vspace{1cm}

\noindent
Permitted aids: Pen and paper. 
% English-Swedish or English-other language dictionary.

There are 6 questions: each worth 10 points. The total sum is 60
points.  

Some questions come with \emph{remarks}: you must take those
into account.  Some questions come with \emph{hints}: you may ignore
those.

You will be asked to write programs in various paradigms. Choose the
language appropriately in each case, and indicate which you choose at
the beginning of your answer.

\begin{tabular}[p]{cc}
  Paradigm & Acceptable language \\ \hline
  Imperative   & C or (an OO language where you refrain to use Objects) \\
  Object oriented & C++ or Java \\
  Functional & Haskell, ML \\
  Concurrent & Erlang, Concurrent-Haskell \\
  Logic & Prolog, Curry 
\end{tabular}

You may also use pseudo-code ressembling an actual language in the
relevant list. In that case, make sure your code can only be
interpreted in the way you intent (the responsibility lies with
you). In particular, in the case of functional/logic languages,
omitting parentheses is NOT acceptable: \texttt{a b c} is not
acceptable pseudo-code for \texttt{a (b c)}.

  
\textbf{Chalmers:}
24 points is required to pass (grade 3), 36 points is required for
grade 4, and 48 points is required for grade 5. 

\textbf{GU:} 
24 points is required to pass (grade G) and 42 points is
required for grade VG.

\section{Generalities}

Suppose that you are facing the task to translate an application
written in Erlang into an imperative language like C. This can be done
by a sequence of transformations seen in the course.  List a sequence
of transformations seen in the course that, if applied \emph{in the
  order given}, will transform the Erlang program to an imperative
program easy to write in C syntax.

Reminder: Erlang is a functional language equipped with concurrency
features such as channels and processes.

\section{Explicit Stack}

Consider the Ackermann function, defined as follows:

\begin{verbatim}
function a(m,n) 
  if m = 0 
    return n+1
  else if n = 0
    return a(m-1,1)
  else
    return a(m-1,a(m,n-1))
\end{verbatim}

\begin{itemize}
\item Write a version of the Ackermann function that does not use
  recursion, but may use an explicit stack.
\item Is it possible to tail-call optimise some recursive
  calls? List which calls which can be optimised:
  \begin{enumerate}
  \item \texttt{a(m-1,1)}
  \item \texttt{a(m-1,a(m,n-1))}
  \item \texttt{a(m,n-1)}
  \end{enumerate}
\end{itemize}

Hints: you should first define the type of values that you can push on
the stack, and you can assume that labels can be used as values.

Note: Optimising all possible tail-calls is worth only 2 points.

\newpage
\section{Objects from records}

Consider the following C++ code, which is an encoding of a class
hierarchy.
\begin{verbatim}
struct C {
   float x;
   float y;
};

struct B {
   void *m(B*, B*);
   int f;
   void *n(B*, C*);
   int g;
};

struct A {
   void *m(A*, B*);
   int f;
};

void B_n(B* p, C* q) {
  q->x += p->f;
  q->y += p->g;
}

void B_m (B* p, B* q) {
  q->f += p->f;
  q->g += p->g;
}

void A_m(A* p, B* q) {
  q->f += p->f;
}
\end{verbatim}

Write the original class hierarchy in a C++-like or Java-like language.

Quick refresher:
\begin{itemize}
\item If \texttt{A} is a type, then \texttt{A*} is the type of
  pointers to \texttt{A}.
\item \texttt{void *f(A,B,...)} stands for a pointer to a function
  \texttt{f} taking arguments \texttt{A},\texttt{B},...
\item \texttt{p->f} is used to access a field \texttt{f} in a
  structure pointed by \texttt{p}.
\end{itemize}
\newpage
\section{Closures}
Consider the Haskell program:

\begin{verbatim}
g [] = []
g (p:ps) = p : g (filter (\x -> x `mod` p == 0) ps)

filter f [] = []
filter f (x:xs) = if p x 
                     then x : filter f xs  
                     else filter f xs
\end{verbatim}

Transform the code to use explicit closures. That is, use a Haskell
\emph{data structure} to represent $\lambda$-expressions.

Remarks:
\begin{itemize}
\item All higher-order functions must be removed.
\item You cannnot \emph{specialize} any higher-order function
  (eg. don't make a special version of \texttt{filter} which can test
  only divisibility).  Use closures.
\end{itemize}
\section{Continuations}

Consider the following function, which computes the fibbonaci number
of its argument.
\begin{verbatim}
fib 0 = return 1
fib n = do x1 <- fib (n-1)
           yield
           x2 <- fib (n-2)
           return (x1 + x2)
\end{verbatim}

The above algorithm will be used on very big numbers, hence
potentially running for a long time, concurrently with other
processes. Therefore, the implementer of \texttt{fib} has decided to
insert a call to \texttt{yield}, which performs no useful
computational task, but gives the opportunity for the runtime
environment to schedule another processes. (The computation of the
fibbonaci number will be continued later.)

Transform \texttt{fib} to use explicit continuations. Remarks: 
\begin{itemize}
\item In the translation, you will use a different version of
  \texttt{yield}, which takes an explicit continuation as argument.
\item The translation of \texttt{fib} should also take an explicit
  continuation as argument.
\item Do not change the algorithm. Do not ``optimize'' it.
\end{itemize}

\newpage
\section{Relations to lists of successes}

Consider the Curry program
\begin{verbatim}
ancestor a c = (parent a x & ancestor x c) | a =:= c
\end{verbatim}



Convert the program from relational to functional style. That is,
write a function \texttt{ancestors} that returns all the ancestors of
a given person.  More precisely, given a person \texttt{c}, construct the
list of \texttt{x}s such that \texttt{ancestor x c} is true.

To do so, you may assume that you have at your disposal a function
\texttt{parents} which returns the list of \texttt{x}s such that
\texttt{parent x c} is true.

Hints: 
\begin{itemize}
\item Remember that \texttt{|} stands for logical disjunction and
  \texttt{\&} stands for logical conjunction.
\item Begin by writing the types of the functions \texttt{ancestors}
  and \texttt{parents}.
\end{itemize}

\end{document}
