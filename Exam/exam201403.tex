\documentclass{article}
\usepackage{graphicx}
\usepackage{amsmath}
\usepackage[utf8]{inputenc}
\usepackage{minted}

\setlength{\parindent}{0pt}
\setlength{\parskip}{6pt plus 2pt minus 1pt}

\usepackage{pgf, tikz}
\usetikzlibrary{shapes}
\usetikzlibrary{shapes.multipart}
\usetikzlibrary{arrows}


\newcommand{\Blue}[1]{\color{blue}#1\color{black}\xspace}

\usepackage{array}
% This is needed because raggedright in table elements redefines \\:
\newcommand{\PreserveBackslash}[1]{\let\temp=\\#1\let\\=\temp}
\let\PBS=\PreserveBackslash

%%%%%%%%%%%%%%
\newcommand{\solution}[1] {}
%\newcommand{\solution}[1] {\textbf{Solutions:}\\ #1}

%\newcommand{\comment}[1]{}
\newcommand{\answer}[0]{\paragraph{Answer:}}
% \newcommand{\answer}[1]{}
\newcommand{\comment}[1]{\marginpar{#1}}


\begin{document}

\setlength{\parskip}{2pt}

\newcommand{\examtime}{8:30, March 15th, 2014  (Location:Maskin)}
\newcommand{\points}[1]{\marginpar{\bf #1 points}}
\noindent
\begin{tabular}{lr}
CHALMERS TEKNISKA H\"OGSKOLA &\examtime{}.\\
Dept. of Computer Science and Engineering & Programming Paradigms\\
Jean-Philippe Bernardy                 & DAT121 / DIT331(GU) \\
\end{tabular}

\vspace{2.5cm} \noindent
\begin{center} {\LARGE
Exam in Programming Paradigms}
\end{center}

\vspace{1.5cm}

\noindent
\examtime{}.\\
\begin{tabular}{ll}
\textbf{Examiner:} & Jean-Philippe Bernardy
\end{tabular}
\vspace{1cm}

\noindent
Permitted aids: Pen and blank paper.
% English-Swedish or English-other language dictionary.

There are 5 questions, each worth 12 points. The total sum is 60
points.

Some questions come with \emph{remarks}: you must take those
into account.
Some questions come with \emph{hints}: even though they are meant to help you, you may ignore those.

You will be asked to write programs in various paradigms. Choose the
language appropriately in each case, and indicate which you choose at
the beginning of your answer.

\begin{tabular}[p]{cc}
  Paradigm & Acceptable language \\ \hline
  Imperative   & C or (an OO language where you refrain to use Objects) \\
  Object oriented & C++ or Java \\
  Functional & Haskell, ML \\
  Concurrent & Erlang, Concurrent-Haskell \\
  Logic & Prolog, Curry
\end{tabular}

You may also use pseudo-code resembling an actual language in the
relevant list. In that case, make sure your code can only be
interpreted in the way you intent (the responsibility lies with
you). In particular, in the case of functional/logic languages,
omitting necessary parentheses is NOT acceptable: \texttt{a b c} means
\texttt{(a b) c} and is not acceptable pseudo-code for \texttt{a (b
  c)}.


\textbf{Chalmers:}
24 points is required to pass (grade 3), 36 points is required for
grade 4, and 48 points is required for grade 5.

\textbf{GU:}
24 points is required to pass (grade G) and 42 points is
required for grade VG.

\section{Explicit gotos}

Consider the following optimised code for array copy:
\begin{verbatim}
  short *to, *from;
  int count;
  ...
  {
     /* pre: count > 0 */
     int n = (count + 7) / 8;
     switch(count % 8){
       case 0:	do{     *to++ = *from++;
       case 7:	        *to++ = *from++;
       case 6:	        *to++ = *from++;
       case 5:	        *to++ = *from++;
       case 4:	        *to++ = *from++;
       case 3:	        *to++ = *from++;
       case 2:	        *to++ = *from++;
       case 1:	        *to++ = *from++;
                } while (--n > 0);
     }
  }
\end{verbatim}

Translate the above snippet to equivalent code that does not use
\texttt{switch} nor \texttt{while}: use explicit gotos instead.
Write your answer in a C-like language.

Remark: you must retain the optimisation. Do not test the loop
condition after each copy of a byte.

Note: You can get partial credit for removing either the switch or the
while.
\answer{See exercise sheets. (All.pdf)}
\newpage
\section{Objects as records}

Consider the following class hierarchy written in a Java-like
language:
\begin{minted}{java}
class A {
  float x;
  float y;
};

class B {
  int n;
  // this method is overridable (the default in Java)
  // in C++ it would be marked 'virtual'
  void f(A p) {
    n = n+n;
  };

};

class C extends B {
  int m;
  // the following method overrides f from class B
  void f(A p) {
    n = n+n;
    m = m+n;
  }
};
\end{minted}

Translate the above classes to records with explicit method pointers.
Remark: remember to write the code which initialises the explicit
method pointers. 

\answer{
\begin{minted}{c}
    struct A {
      float x = 0
      float y = 0
    }
    struct B {
      int n = 0
      void* f(B* this, A* p) = <BODY1>;
    }
    struct C {
      int n = 0
      void* f(C* this, A* p) = <BODY2>;
      int m;
    }
\end{minted}
}

\newpage
\section{Pattern matching and Higher-Order Abstractions}

Consider the following code:
\begin{minted}{haskell}

foobar :: [Int] -> Int
foobar []     = 0
foobar (x:xs)
  | x > 20     = (5*x-20) + foobar xs
  | otherwise = foobar xs

sum [] = 0
sum (x:xs) = x + sum xs

map f [] = []
map f (x:xs) = (f x) : (map f xs)

filter f [] = []
filter f (x:xs) = if f x then x:(filter f xs) else filter f xs

fold k f [] = k
fold k f (x:xs) = f x (fold k f xs)
\end{minted}

\begin{itemize}
  \item Express the function \texttt{foobar} in terms of \texttt{sum}, \texttt{map} and \texttt{filter}.\points {6}
  \item Express the function \texttt{foobar} in terms of \texttt{fold} ONLY.\points {6}
\end{itemize}

Remark: In particular you may NOT use recursion (directly) in your definitions.

\answer{
  \begin{minted}{haskell}
  foobar = sum . map (\x -> 7*x+2) . filter (>3)
  foobar = fold 0 f where
     f x acc = if x > 3 then 7*x + 2 + acc else acc
   \end{minted}
}
\newpage
\section{Continuations and Closures}

Replace lambda abstractions by explicit closures in the following
code.
\begin{minted}{haskell}
f # x = f x

data Tree a = Leaf | Bin (Tree a) a (Tree a)

traverseC :: Tree Int -> (Int -> Int) -> Int
traverseC Leaf k = k 0
traverseC (Bin l x r) k =
  traverseC l # \tl ->
  traverseC r # \tr ->
  k (tl + x + tr)

traverse t = traverseC t (\x -> x)
\end{minted}

\answer{
  \begin{minted}{haskell}
data Closure = Id
             | K1 (Tree Int) Int Closure
             | K2 Int Int Closure

ap :: Closure -> Int -> Int
ap Id x = x
ap (K2 tl x k) tr = ap k (tl + x + tr)
ap (K1 r x k) tl = traverseC r # K2 tl x k

traverseC Leaf k = ap k 0
traverseC (Bin l x r) k = traverseC l # K1

traverse t = traverseC t Id
  \end{minted}
}
\newpage
\section{Unification}

The following implementation of the unification algorithm contains two mistakes. Hint: they are within the code of 'unify'.
\begin{itemize}
\item In which lines(s) of the unify function do the mistake occur?
  \points{4}
\item Explain each of the mistakes. \points{4}
\item Write a correct version of the algorithm. You need only write
  the lines that need to be changed. Be explicit about where the
  correct code should go or what it should replace. \points{4}
\end{itemize}



\begin{minted}{haskell}
    both f (x,y) = (f x, f y)
    data Term = Con String [Term] -- the terms are the arguments to the constant
              | Var Int -- metavariable
      deriving Eq
(1) unify :: [(Term,Term)] -> Substitution -> Maybe Substitution
(2) unify [] s = Just s -- Base case
(3) unify ((t,t'):ts) s | t == t' = unify ts s
(4) unify ((Con f as,Con g bs):ts) s
(5)   | length as == length bs = unify (zip as bs ++ ts) s
(6)   | otherwise = Nothing -- Clash
(7) unify ((Var x,t):ts) s = unify (map (both (applySubst (x ==> t))) ts) (s +> (x,t))
(8) unify ((t, Var x):ts) s = unify ((Var x, t):ts) s -- Re-orient
\end{minted}


For completeness here is the code for management of substitutions.

\begin{minted}{haskell}
type Substitution = Map Int Term

-- | Identity (nothing is substituted)
idSubst = M.empty

-- | Add an "assignment" to a substitution
(+>) :: Substitution -> (Int, Term) -> Substitution
s +> (x,t) = M.insert x t (M.map (applySubst (x ==> t)) s)

-- | Single substitution
(==>) :: Int -> Term -> Substitution
x ==> t = M.singleton x t

-- | Apply a substitution to a term
applySubst :: Substitution -> Term -> Term
applySubst f (Var i) = case M.lookup i f of
                          Nothing -> Var i
                          Just t -> t
applySubst f (Con c xs) = Con c (map (applySubst f) xs)
\end{minted}

\answer{
  \begin{itemize}
  \item The errors are in lines 4 and 7.
  \item 4. One needs to check equality of f and g
        7. One must add the occurs check.
  \item if (x `elems` varsOf t) then Nothing else ...
  \end{itemize}
}

\end{document}
