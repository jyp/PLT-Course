\documentclass{article}
\usepackage{graphicx}
\usepackage{amsmath}
\usepackage[utf8]{inputenc}
\usepackage{minted}

\setlength{\parindent}{0pt}
\setlength{\parskip}{6pt plus 2pt minus 1pt}

\usepackage{pgf, tikz}
\usetikzlibrary{shapes}
\usetikzlibrary{shapes.multipart}
\usetikzlibrary{arrows}


\newcommand{\Blue}[1]{\color{blue}#1\color{black}\xspace}

\usepackage{array}
% This is needed because raggedright in table elements redefines \\:
\newcommand{\PreserveBackslash}[1]{\let\temp=\\#1\let\\=\temp}
\let\PBS=\PreserveBackslash

%%%%%%%%%%%%%%
\newcommand{\solution}[1] {}
%\newcommand{\solution}[1] {\textbf{Solutions:}\\ #1}

%\newcommand{\comment}[1]{}
\newcommand{\answer}{\paragraph{Answer:}}
\newcommand{\comment}[1]{\marginpar{#1}}


\begin{document}

\setlength{\parskip}{2pt}

\newcommand{\examtime}{14:00, March 14th, 2013}
\newcommand{\points}[1]{\marginpar{\bf #1 points}}
\noindent
\begin{tabular}{lr}
CHALMERS TEKNISKA H\"OGSKOLA &\examtime{}.\\
Dept. of Computer Science and Engineering & Programming Paradigms\\
Jean-Philippe Bernardy                 & DAT121 / DIT331(GU) \\
\end{tabular}

\vspace{2.5cm} \noindent
\begin{center} {\LARGE
Exam in Programming Paradigms}
\end{center}

\vspace{1.5cm}

\noindent
\examtime{}.\\
\begin{tabular}{ll}
\textbf{Examiner:} & Jean-Philippe Bernardy
\end{tabular}
\vspace{1cm}

\noindent
Permitted aids: Pen and paper.
% English-Swedish or English-other language dictionary.

There are 6 questions: each worth 10 points. The total sum is 60
points.

Some questions come with \emph{remarks}: you must take those
into account.  
Some questions come with \emph{hints}: you may ignore those.

You will be asked to write programs in various paradigms. Choose the
language appropriately in each case, and indicate which you choose at
the beginning of your answer.

\begin{tabular}[p]{cc}
  Paradigm & Acceptable language \\ \hline
  Imperative   & C or (an OO language where you refrain to use Objects) \\
  Object oriented & C++ or Java \\
  Functional & Haskell, ML \\
  Concurrent & Erlang, Concurrent-Haskell \\
  Logic & Prolog, Curry
\end{tabular}

You may also use pseudo-code resembling an actual language in the
relevant list. In that case, make sure your code can only be
interpreted in the way you intent (the responsibility lies with
you). In particular, in the case of functional/logic languages,
omitting parentheses is NOT acceptable: \texttt{a b c} is not
acceptable pseudo-code for \texttt{a (b c)}.


\textbf{Chalmers:}
24 points is required to pass (grade 3), 36 points is required for
grade 4, and 48 points is required for grade 5.

\textbf{GU:}
24 points is required to pass (grade G) and 42 points is
required for grade VG.

\section{Objects as records}

Consider the following class hierarchy written in a Java-like
language:
\begin{minted}{java}
class A {
  float x;
  float y;
};

class B {
  int n;
  void f(A p) {
    <BODY 1>
  };

};

class C extends B {
  int m;
  // the following method overrides f from class B
  void f(A p) { 
    <BODY 2>
  }
};
\end{minted}

Translate the above classes to records with explicit method pointers.

Demonstrate your solution by drawing a diagram of a memory area
corresponding to an object of type...
\begin{itemize}
\item \texttt A. \points{1}
\item \texttt B. \points{3}
\item \texttt C. \points{6}
\end{itemize}
Remarks: make sure all the fields are initialized to some value. The
values can be arbitrary but must be consistent with the idea of the
translation to records with explicit method pointers. If your solution
contains pointers, you must show what they point to.

\answer{
\begin{minted}{c}
    struct A {
      float x = 0
      float y = 0
    }
    struct B {
      int n = 0
      void* f(B* this, A* p) = <BODY1>;
    }
    struct C {
      int n = 0
      void* f(C* this, A* p) = <BODY2>;
      int m;
    }
\end{minted}
}

\newpage
\section{Calls and recursion}

Consider the following function \texttt{hanoi},
which solves the Hanoi tower problem.

\begin{minted}{c}
void hanoi(int n,int s,int i,int d) {
  if (n>0) {
    hanoi(n-1,s,d,i);
    move(s,d);
    hanoi(n-1,i,s,d);
  }
}
\end{minted}

Solve the following two exercises independently.
\begin{enumerate}
\item Replace the last recursive call by a while loop (tail-call
  elimination). Remark: do not use a stack. \points 4

\answer{
\begin{minted}{c}
    void move_many(int n, int source, int intermediate, int target) {
      while (n > 0) {
        move_many(n-1,source,target,intermediate);
        move_disk(source,target);
        n = n-1;
        swap(source,intermediate);
      }
    }
\end{minted}
}
\item Assume a stack; with the usual push, pop and top
  functions. Transform the original \texttt{hanoi} function by making
  the calls explicit. Remarks: use gotos, and stack to transform \emph{each}
  recursive call. Transform the original \texttt{hanoi} function, not
  the one obtained in the previous exercise. \points 6
  \answer{
    \begin{minted}{c}
      void move_many(int n, int s, int i, int d) {
        push(n,s,i,d,0);
      call:
        if (stk->n!=0) {
          push(stk->n-1,stk->s,stk->i,stk->d,1);
          goto call;
      loc1:
          move_disk(stk->s,stk->d);
          push(stk->n-1,stk->i,stk->s,stk->d,2);;
          goto call;
      loc2:
        }
        caller = stk->caller;
        pop();
        if (caller == 1) goto loc1;
        if (caller == 2) goto loc2;
      }
    \end{minted}
  }
\end{enumerate}
\newpage
\section{Pattern matching Higher-Order Abstractions}

Assume the following Haskell data type:
\begin{minted}{haskell}
data List = Nil | Cons Int List
\end{minted}
and consider the following code:
\begin{minted}{haskell}
map f Nil = Nil
map f (Cons x xs) = Cons (f x) (map f xs)

append Nil ys = ys
append (Cons x xs) ys = Cons x (append xs ys)

reverse Nil = Nil
reverse (Cons x xs) = append (reverse xs) (Cons x Nil)
\end{minted}

Express each of the functions above (\texttt{map}, \texttt{append},
\texttt{reverse}) \points {3,3,4} in terms of the higher-order
function \texttt{fold} provided below.  Remarks: you may \emph{not}
use recursion directly in your solution. The type of the functions
should remain the same.

\begin{minted}{haskell}
fold k f Nil = k
fold k f (Cons x xs) = f x (fold k f xs)
\end{minted}
\answer{
  \begin{minted}{haskell}
    map f = fold Nil (\x -> Cons (f x))
    append xs ys = fold ys Cons xs
    reverse = fold Nil (\x xs -> append xs (Cons x Nil))
  \end{minted}
}
\newpage
\section{Closures}

Consider the following partial implementation of the ``Observer''
pattern in a Java-like language:
\begin{minted}{java}
interface Observer {
  bool notify(Event e);
}

class MyPrintAction implements Observer {
   String text;

   bool notify(Event e) {
     print("Got the event:" + e);
     print(text);
     return true;
   }
}

class Button {
  List<Observer> os;

  void onPress (Observer l) {
    os.add(l);
  }
}
\end{minted}

And the example user code:

\begin{minted}{java}
button.onPress(new MyPrintAction("Button pressed."));
\end{minted}

Assuming that the interface \texttt{Observer} represents a closure, how
could this pattern be implemented naturally in a language with native
support for higher-order functions?

Demonstrate your answer by translating the pattern to Haskell as
follows:
\begin{enumerate}
\item Give the Haskell type corresponding to the Observer interface.\points 3
\item Give the type of the translated onPress function. (A Haskell type) \points 2
\item Translate the MyPrintAction class. \points 2
\item Translate the example call to onPress. \points 3
\end{enumerate}
Hints: You may assume a Haskell function \texttt{print} which prints
objects of any type.  You may (if you prefer) answer questions 3. and
4. in a single piece of code.

\answer{
\begin{minted}{haskell}
  type Observer = Event -> IO Bool
  
  onPress :: Button -> Observer -> IO ()
  
  myPrintAction text e = do
    print ("Got the event" ++ show e)
    print text

  example = onPress button (myPrintAction "button pressed");
\end{minted}
}
\newpage
\section{Variable-managing process}

Assume a language without pointers nor variables, but support for
concurrency. In particular, assume primitives for creating processes
and channels, and primitives for reading and writing to to
channels. For example C++-style:

\begin{minted}{cpp}
Chan<A> newChan();
A readChan(Chan<A> c);
void writeChan(Chan<A> c,A x);
void forkProcess(*void());
\end{minted}
or Haskell-style:
\begin{minted}{haskell}
newChan :: IO (Chan a)
readChan :: Chan a -> IO a
writeChan :: Chan a -> a -> IO ()
forkProcess :: IO () -> IO ()
\end{minted}

Your task is to simulate references to mutable variables using the
above primitives. This can be done be using a process that manages the
variable state.

\begin{enumerate}
\item Define the representation for a variable of type ``reference to
  Integer''. Name this type \texttt{Reference}. \points 3
\item Write the code for the process that manages the variable state. \points 4
\item Write the code for primitives to create, read and write references. \points 3
Their type should be:
\begin{minted}{cpp}
Reference newRef();
int readRef(Reference);
void writeRef(Reference, int);
\end{minted}
or Haskell-style:
\begin{minted}{haskell}
newRef :: IO Reference
readRef :: Reference -> IO Int
writeRef :: Reference -> Int -> IO ()
\end{minted}
\end{enumerate}

Hint: The channels can transmit any type of information, including
references to channels.

Remark: You can use either a functional or imperative language to
write your answer, however you may not use any global variable nor
primitive reference types in it.

\answer{
\begin{minted}{haskell}    
data Command a = Get (Chan a) | Set a
type Variable a = Chan (Command a)

handler :: Variable a -> a -> IO ()
handler v a = do
  command <- readChan v
  case command of
    Set a' -> handler v a'
    Get c -> do 
      writeChan c a
      handler v a

newVariable :: a -> IO (Variable a)
newVariable a = do
  c <- newChan
  forkIO (handler c a)
  return c

get :: Variable a -> IO a
get v = do
  c <- newChan
  writeChan v (Get c)
  readChan c
  
set :: Variable a -> a -> IO ()
set v a = do
  writeChan v (Set a)
\end{minted}
}

\newpage
\section{Relations}

Consider the following Haskell code:

\begin{minted}{haskell}
data List = Nil | Cons Int List
data Tree = Tip | Bin Tree Int Tree

append :: List -> List -> List
append Nil xs = xs
append (Cons x xs) ys = Cons x (append xs ys)

flatten :: Tree -> List
flatten Tip = Nil
flatten (Bin a x b) = append (flatten a) (Cons x (flatten b))
\end{minted}

Translate the above functions to relations \texttt{flatten'} and
\texttt{append'} such that:
\begin{minted}{haskell}
       append' x y z       is equivalent to      append x y == z
       flatten' x y        is equivalent to      flatten x == y
\end{minted}

\begin{enumerate}
\item Write the type of the relations \texttt{flatten'} and
  \texttt{append'}. \points 4
\item Write their code. \points 6
\end{enumerate}

Remark: you cannot use any other helper function in your answer, only
constructors and relations. The types should be written in Curry or
Haskell-like syntax.

You may write the code in Curry syntax or Prolog syntax. 
\answer{
\begin{minted}{haskell}
    append :: List -> List -> List -> Success
    append []     ys zs = ys =:= zs
    append (x:xs) ys zs = append xs ys zs' &
                          zs =:= x:zs'
       where zs' free

   flatten :: Tree -> List -> Success
   flatten Tip Nil = success
   flatten (Bin a x b) zs = flatten a xs & flatten b ys & append xs (Cons x ys) zs
      where xs, ys free
  \end{minted}
}
\end{document}
