\documentclass{article}
\usepackage{graphicx}
\usepackage{amsmath}
\usepackage[mathletters]{ucs}
\usepackage[utf8x]{inputenc}
\usepackage{listings}
\lstnewenvironment{code}{\lstset{language=Haskell,basicstyle=\small\ttfamily}}{}

\setlength{\parindent}{0pt}
\setlength{\parskip}{6pt plus 2pt minus 1pt}

\usepackage{pgf, tikz}
\usetikzlibrary{shapes}
\usetikzlibrary{shapes.multipart}
\usetikzlibrary{arrows}

\lstdefinelanguage{FSharp}
      {morekeywords={let, new, match, inherit, abstract, with, rec,
          open, module, namespace, type, of, member, and, for, in, do,
          begin, end, fun, function, try, mutable, if, then, else,
          class, interface, end},
    keywordstyle=\color{blue},
    basicstyle = \small,
    sensitive=false,
    morecomment=[l][\color{green!50!black!80}]{///},
    morecomment=[l][\color{green!50!black!80}]{//},
    morecomment=[s][\color{green!50!black!80}]{{(*}{*)}},
    morestring=[b]",
    stringstyle=\color{red},
    columns=fullflexible
    }


\lstdefinelanguage{Smalltalk}{
  basicstyle=\ttfamily,
  keywordstyle=\bfseries,
  morekeywords={self,super,true,false,nil,thisContext}, % This is overkill
  morestring=[d]',
  morecomment=[s]{"}{"},
  alsoletter={\#:},
  escapechar={!},
}[keywords,comments,strings]


\newcommand{\Blue}[1]{\color{blue}#1\color{black}\xspace}


\usepackage{array}
% This is needed because raggedright in table elements redefines \\:
\newcommand{\PreserveBackslash}[1]{\let\temp=\\#1\let\\=\temp}
\let\PBS=\PreserveBackslash

%%%%%%%%%%%%%% 
\newcommand{\solution}[1] {}
%\newcommand{\solution}[1] {\textbf{Solutions:}\\ #1}

%\newcommand{\comment}[1]{}
\newcommand{\comment}[1]{\marginpar{#1}}


\begin{document}
\newcommand{\examtime}{14:00, March 9th, 2012}
\newcommand{\points}[1]{\marginpar{\bf #1 points}}
\noindent
\begin{tabular}{lr}
CHALMERS TEKNISKA H\"OGSKOLA &\examtime{}.\\
Dept. of Computer Science and Engineering & Programming Paradigms\\
Jean-Philippe Bernardy                 & DAT121 / DIT331(GU) \\
\end{tabular}

\vspace{2.5cm} \noindent
\begin{center} {\LARGE
Exam in Programming Paradigms}
\end{center}

\vspace{1.5cm}

\noindent
\examtime{}.\\
\begin{tabular}{ll}
\textbf{Examiner:} &  John Hughes  \\
\textbf{Lecturer:} & Jean-Philippe Bernardy
\end{tabular}
\vspace{1cm}

\noindent
Permitted aids:\\
English-Swedish or English-other language dictionary.

There are 6 questions: each worth 10 points. The total sum is 60
points.  Some questions come with \emph{remarks}: you must take those
into account.  Some questions come with \emph{hints}: you may ignore
those.


\textbf{Chalmers:}
24 points is required to pass (grade 3), 36 points is required for
grade 4, and 48 points is required for grade 5. 

\textbf{GU:} 
24 points is required to pass (grade G) and 42 points is
required for grade VG.


\newpage
\hfill\\
\newpage


\section{Explicit gotos}

Consider the following optimised code for array copy:
\begin{verbatim}
  short *to, *from;
  int count;
  ...
  {
     /* pre: count > 0 */
     int n = (count + 7) / 8;
     switch(count % 8){
       case 0:	do{     *to++ = *from++;
       case 7:	        *to++ = *from++;
       case 6:	        *to++ = *from++;
       case 5:	        *to++ = *from++;
       case 4:	        *to++ = *from++;
       case 3:	        *to++ = *from++;
       case 2:	        *to++ = *from++;
       case 1:	        *to++ = *from++;
                } while (--n > 0);
     }
  }
\end{verbatim}

Translate the above snippet to equivalent code that does not use
\texttt{switch} nor \texttt{while}: use explicit gotos instead.

Remark: you must retain the optimisation. Do not test the loop
condition after each copy of a byte.

\newpage
\section{Objects from records}

Consider the following code, which is an encoding of a class
hierarchy, written in C.
\begin{verbatim}

struct C {
   float x;
   float y;
};

struct C {
   void *n(B *p, C* q);
   void *m(B *p, B *q);
   int f;
   int g;
};

struct A {
   void *m(A *p, B *q);
   int f;
}

void B_n(B* p, C* q) {
  q->x += p->f;
  q->y += p->g;
}

void B_m (B* p, B* q) {
  q->f += p->f;
  q->g += p->g;
}

void A_m(A* p, B* q) {
  q->f += p->f;
}
\end{verbatim}

Write the original class hierarchy in a C++-like or Java-like language.


\newpage
\section{Higher-order abstractions}

Consider the following code:
\begin{verbatim}
concat [] = []
concat (x:xs) = x ++ concat xs
\end{verbatim}

Express the same function in terms of the higher-order function
\texttt{fold} provided below.  Remark: you may \emph{not} use
recursion directly in your solution.

\begin{verbatim}
fold k f [] = k
fold k f (x:xs) = f x (fold k f xs)
\end{verbatim}

\section{Closures}

Consider the program:
\begin{verbatim}
f grades bonuses = zipWith (+) bonuses (filter (<48) grades)

zipWith f [] [] = []
zipWith f (x:xs) (y:xs) = f x y : zipWith f xs ys

filter p [] = []
filter p (x:xs) = if p x
                  then x : filter p xs
                  else filter p xs
\end{verbatim}

Translate the code to use explicit closures. 

Remarks:
\begin{itemize}
\item All higher-order functions must be removed.
\item You may not \emph{specialize} any higher-order function to the
  cases where they are used here. Use closures!
\end{itemize}

\newpage
\section{Laziness}

In Haskell you can write a matrix comprehension like so:
\begin{verbatim}
matrix = array bounds [((x,y),valueAtIndex x y) | (x,y) <- range bounds]
  where bounds =  ((lowx,highx),(lowy,highy))
\end{verbatim}

You can then access a cell $x,y$ in the array using $matrix!(x,y)$.

For example, in this array the value at index $(x,y)$ contains $x ×
y$: \texttt{example!(x,y) == x * y}.
\begin{verbatim}
example = array bounds [((x,y),x * y) | (x,y) <- range bounds]
  where bounds =  ((0,0),(10,10))
\end{verbatim}


Consider the following algorithm to compute the Levenshtein distance between two strings:
(see next page)
\newpage
\begin{verbatim}
int LevenshteinDistance(char s[1..m], char t[1..n])
{
  // for all i and j, d[i,j] will hold the Levenshtein distance between
  // the first i characters of s and the first j characters of t;
  // note that d has (m+1)x(n+1) values
  declare int d[0..m, 0..n]

  for i from 0 to m
    d[i, 0] := i // the distance of any first string to an empty second string
  for j from 0 to n
    d[0, j] := j // the distance of any second string to an empty first string

  for j from 1 to n
  {
    for i from 1 to m
    {
      if s[i] = t[j] then
        d[i, j] := d[i-1, j-1]       // no operation required
      else
        d[i, j] := minimum
                   (
                     d[i-1, j] + 1,  // a deletion
                     d[i, j-1] + 1,  // an insertion
                     d[i-1, j-1] + 1 // a substitution
                   )
    }
  }

  return d[m,n]
}
\end{verbatim}


Using \emph{a single comprehension}, construct a matrix \texttt{d},
such that \texttt{d!(i,j)} is the Levenshtein distance between the
first \texttt{i} characters of \texttt{s} and the first \texttt{j}
characters of \texttt{t}, assuming
\begin{verbatim}
s :: String
t :: String
\end{verbatim}

Hints:
\begin{itemize}
\item You may also assume a predefined function \texttt{minimum}:
\begin{verbatim}
   minimum :: [Int] -> Int
\end{verbatim}
\item You can access the $\mbox{\texttt{i}}^{\mbox{th}}$ character of
  string \texttt{s} with the expression \texttt{s!!(i-1)}.
\end{itemize}

\newpage
\section{Functions to relations}

Consider the function
\begin{verbatim}
fib :: Int -> Int
fib 0 = 0
fib 1 = 1
fib n = fib (n-1) + fib (n-2)
\end{verbatim}

Translate the function \texttt{fib} to a relation \texttt{fibo}, such that
\begin{verbatim}
       fibo x y z     is equivalent to      fib x y == z
\end{verbatim}

Remarks:
\begin{itemize}
\item You may not use the \emph{function} \texttt{fib} in the
  translated version.
\item Use a relation \texttt{plus} which coincides with addition:
  \begin{verbatim}
       plus x y z     is equivalent to      x + y == z
\end{verbatim}

\begin{verbatim}

\end{verbatim}
\end{itemize}

Hint: you may want to start by writing the types of the
\emph{relations} \texttt{fib} and \texttt{plus}.

\end{document}
