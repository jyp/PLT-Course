\documentclass{article}
\usepackage{graphicx}
\usepackage{amsmath}
\usepackage[utf8]{inputenc}
\usepackage{minted}

\setlength{\parindent}{0pt}
\setlength{\parskip}{6pt plus 2pt minus 1pt}

\usepackage{pgf, tikz}
\usetikzlibrary{shapes}
\usetikzlibrary{shapes.multipart}
\usetikzlibrary{arrows}


\newcommand{\Blue}[1]{\color{blue}#1\color{black}\xspace}

\usepackage{array}
% This is needed because raggedright in table elements redefines \\:
\newcommand{\PreserveBackslash}[1]{\let\temp=\\#1\let\\=\temp}
\let\PBS=\PreserveBackslash

%%%%%%%%%%%%%%
\newcommand{\solution}[1] {}
%\newcommand{\solution}[1] {\textbf{Solutions:}\\ #1}

%\newcommand{\comment}[1]{}
\newcommand{\answer}[0]{\paragraph{Answer:}}
% \newcommand{\answer}[1]{}
\newcommand{\comment}[1]{\marginpar{#1}}


\begin{document}

\setlength{\parskip}{2pt}

\newcommand{\examtime}{Afternoon, August 28th  (Location:M)}
\newcommand{\points}[1]{\marginpar{\bf #1 points}}
\noindent
\begin{tabular}{lr}
CHALMERS TEKNISKA H\"OGSKOLA &\examtime{}.\\
Dept. of Computer Science and Engineering & Programming Paradigms\\
Jean-Philippe Bernardy                 & DAT121 / DIT331(GU) \\
\end{tabular}

\vspace{2.5cm} \noindent
\begin{center} {\LARGE
Exam in Programming Paradigms}
\end{center}

\vspace{1.5cm}

\noindent
\examtime{}.\\
\begin{tabular}{ll}
\textbf{Examiner:} & Jean-Philippe Bernardy \\
\textbf{Permitted aids:} & Pen and blank paper.
\end{tabular}
\vspace{1cm}

\noindent
% English-Swedish or English-other language dictionary.

There are 5 questions, each worth 12 points. The total sum is 60
points.

Some questions come with \emph{remarks}: you must take those
into account.
Some questions come with \emph{hints}: even though they are meant to help you, you may ignore those.

You will be asked to write programs in various paradigms. Choose the
language appropriately in each case, and indicate which you choose at
the beginning of your answer.

\begin{tabular}[p]{cc}
  Paradigm & Acceptable language \\ \hline
  Imperative   & C or (an OO language where you refrain to use Objects) \\
  Object oriented & C++ or Java \\
  Functional & Haskell, ML \\
  Concurrent & Erlang, Concurrent-Haskell \\
  Logic & Prolog, Curry
\end{tabular}

You may also use pseudo-code resembling an actual language in the
relevant list. In that case, make sure your code can only be
interpreted in the way you intent (the responsibility lies with
you). In particular, in the case of functional/logic languages,
omitting necessary parentheses is NOT acceptable: \texttt{a b c} means
\texttt{(a b) c} and is not acceptable pseudo-code for \texttt{a (b
  c)}.


\textbf{Chalmers:}
24 points is required to pass (grade 3), 36 points is required for
grade 4, and 48 points is required for grade 5.

\textbf{GU:}
24 points is required to pass (grade G) and 42 points is
required for grade VG.

\section{Interpreter of functions}

\section{Objects, Records, Method pointers}
Consider the following class hierarchy written in a Java-like
language:
\begin{minted}{java}
class A {
  float x;
  float y;
};

class B {
  int n;
  // this method is overridable (the default in Java)
  // in C++ it would be marked 'virtual'
  void f(A p) {
    n = n+n;
  };

};

class C extends B {
  int m;
  // the following method overrides f from class B
  void f(A p) {
    n = n+n;
    m = m+n;
  }
};

static void test(A a, B b) {
  B.f(a);
}
\end{minted}

\begin{itemize}
\item Translate the above classes (A,B,C) to records with explicit method
  pointers.  Remark: remember to write the code which initialises the
  explicit method pointers. Assume that the arguments are passed by
  reference.  \points{6}
\item Translate the static method (function) test. Assume that the
  arguments are passed by reference.  \points{3}
\item Translate the static method (function) test. Assume that the
  argument $b$ is passed by value (all the others are still passed by
  reference.)  \points{3}
\end{itemize}

\answer{
\begin{minted}{c}
    struct A {
      float x = 0
      float y = 0
    }
    struct B {
      int n = 0
      void* f(B* this, A* p) = <BODY1>;
    }
    struct C {
      int n = 0
      void* f(C* this, A* p) = <BODY2>;
      int m;
    }
\end{minted}
}

\section{Continuations and Recursion}

Consider the following function, which computes the fibbonaci number
of its argument.
\begin{verbatim}
fib :: Int -> Int
fib 0 = 1
fib 1 = 1
fib n = fib (n-1) + fib (n-2)
\end{verbatim}

\begin{enumerate}
\item How many non-tail calls to fib does the above function contain?
  \points{2}
\item True or false: a tail-call needs stack space to be
  implemented. \points{2}
\item Translate the above function to use explicit continuations.
  \\ Remark: Assume an ambient type of effects called Effect.
  \begin{enumerate}
  \item  Write the type of the translated fib function. \points {2}
  \item  Write the body of the translated fib function. \points {4}
  \end{enumerate}
\item How many non-tail calls to fib does the (correctly) translated function
  contain? \points{2}
\end{enumerate}

\section{Closures}

Consider the following piece of Haskell code.
\begin{minted}{haskell}
data Point = Point Float Float
type Region = Point -> Bool

(.+.) :: Point -> Point -> Point
Point x1 y1 .+. Point x2 y2 = Point (x1 + x2) (y1 + y2)

opposite :: Point -> Point
opposite (Point x y) = Point (negate x) (negate y)

(.-.) :: Point -> Point -> Point
p1 .-. p2 = p1 .+. opposite p2

norm2 :: Point -> Float
norm2 (Point x y) = x*x + y*y

outside :: Region -> Region
outside r = \p -> not (r p)

intersect :: Region -> Region -> Region
intersect r1 r2 p = r1 p && r2 p

withinRange :: Float -> Point -> Region
withinRange range p1 p2 = norm2 (p1 .-. p2) <= range * range
\end{minted}

Transform the above code to use explicit closures for the
\texttt{Region} type.

\begin{itemize}
\item Create a data type containing a constructor for each operation constructing a \texttt{Region} \points{3}
\item Give the type of a function \texttt{apply} which converts each constructor of the \texttt{Region} data type into a function (\texttt{Point -> Bool}). \points{3}
\item Give its definition \points{3}
\item Give the new code for each operation, by referencing to the constructors of the new \texttt{Region} data type \points{3}
\end{itemize}

\answer{
  \begin{minted}{haskell}
data Region = Outside Region
            | Intersect Region Region
            | Within Float Point

apply :: Region -> Point -> Bool
apply (Outside r) p       = not (apply r p)
apply (Intersect r1 r2) p = apply r1 p && apply r2 p
apply (Within range p1) p2 = norm2 (p1 .-. p2) <= range * range

outside :: Region -> Region
outside r = Outside r

intersect :: Region -> Region -> Region
intersect r1 r2 = Intersect r1 r2

withinRange :: Float -> Point -> Region
withinRange range p1 p2 = Within range p1 p2
  \end{minted}
}

\section{State-Management Process}
Assume a language without pointers nor variables, but support for
concurrency. In particular, assume primitives for creating processes
and channels, and primitives for reading and writing to to
channels. For example C++-style:

\begin{minted}{cpp}
Chan<A> newChan();
A readChan(Chan<A> c);
void writeChan(Chan<A> c,A x);
void forkProcess(*void());
\end{minted}
or Haskell-style:
\begin{minted}{haskell}
newChan :: IO (Chan a)
readChan :: Chan a -> IO a
writeChan :: Chan a -> a -> IO ()
forkProcess :: IO () -> IO ()
\end{minted}

Your task is to simulate references to mutable variables using the
above primitives. This can be done be using a process that manages the
variable state.

\begin{enumerate}
\item Define the representation for a variable of type ``reference to
  Integer''. Name this type \texttt{Reference}. \points 3
\item Write the code for the process that manages the variable state. \points 4
\item Write the code for primitives to create, read and write references. \points 3
Their type should be:
\begin{minted}{cpp}
Reference newRef();
int readRef(Reference);
void writeRef(Reference, int);
\end{minted}
or Haskell-style:
\begin{minted}{haskell}
newRef :: IO Reference
readRef :: Reference -> IO Int
writeRef :: Reference -> Int -> IO ()
\end{minted}
\item Explain what happens during a readRef operation. In particular:
  \begin{enumerate}
  \item How many channel(s) are used? \points 1
  \item For each of the channel(s), how many message(s) are sent?  \points 1
  \end{enumerate}
\end{enumerate}

Hint: The channels can transmit any type of information, including
references to channels.

Remark: You can use either a functional or imperative language to
write your answer, however you may not use any global variable nor
primitive reference types in it.

\answer{
\begin{minted}{haskell}    
data Command a = Get (Chan a) | Set a
type Variable a = Chan (Command a)

handler :: Variable a -> a -> IO ()
handler v a = do
  command <- readChan v
  case command of
    Set a' -> handler v a'
    Get c -> do 
      writeChan c a
      handler v a

newVariable :: a -> IO (Variable a)
newVariable a = do
  c <- newChan
  forkIO (handler c a)
  return c

get :: Variable a -> IO a
get v = do
  c <- newChan
  writeChan v (Get c)
  readChan c
  
set :: Variable a -> a -> IO ()
set v a = do
  writeChan v (Set a)
\end{minted}
}

\section{Unification}

\begin{minted}{haskell}
    both f (x,y) = (f x, f y)
    data Term = Con String [Term] -- the terms are the arguments to the constant
              | Var Int -- metavariable
      deriving Eq
(1) unify :: [(Term,Term)] -> Substitution -> Maybe Substitution
(2) unify [] s = Just s -- Base case
(3) unify ((t,t'):ts) s | t == t' = unify ts s
(4) unify ((Con f as,Con g bs):ts) s
(5)   | f == g && length as == length bs = unify (zip as bs ++ ts) s
(6)   | otherwise = Nothing -- Clash
(7) unify ((Var x,t):ts) s = unify (map (both (applySubst (x ==> t))) ts) (s +> (x,t))
(8) unify ((t, Var x):ts) s = unify ((Var x, t):ts) s -- Re-orient
\end{minted}

The above algorithm for unification does not perform the occurs test.

\begin{enumerate}
\item Give an example of a unification constraints which fails if the
  occurs test is enabled \points{3}
\item Explain a solution of your example, assuming the occurs test is
  disabled \points{3}
\item The occurs test can be added by adding a single line in the
  above algorithm. Between which lines should the test be added?
  \points{2}
\item Write the line in question. (You may assume standard
  Term-manipulation functions.) \points{4}
\end{enumerate}

\end{document}
