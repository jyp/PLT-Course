\documentclass{article}



\begin{document}

\section{Imperative Programming [12 points]}

\begin{enumerate}
\item (Parameter passing.) 
\hfill{\textbf{[4 points]}}
\begin{enumerate}
\item Given  the following program:

{\small
\begin{verbatim}
int i;
int a[2];
void p(int x, int y) {
    x++;
    i++;
    y++;
}
int main () {
    a[0] := 1;
    a[1] := 1;
    i := 0;
    p(a[i],a[i]);
    print ("a[0] = ", a[0]);
    print ("a[1] = ", a[1]);
    return 0;
}
\end{verbatim}
}

For each of the following methods of parameter passing,
draw the activation record of the subroutine \texttt{p} at the time when
the parameters are passed. Also give the output of above program
in each of the cases

\begin{enumerate}\itemsep=0.0cm
\item Call-by-reference
\item Call-by-value
\item Call-by-result
\item Call-by-value-result; if the return address is computed...
\begin{enumerate}\itemsep=0.0cm
\item early (beginning of the call)
\item for x an y separately (before and after i++)
\item late (when the call returns)
\end{enumerate}
\end{enumerate}
 (\textbf{2p.})
\item Consider the following definition of the 
subroutine \texttt{sum}: 

{\small 
\begin{verbatim}
int sum (by-name int a, by-name int index, int size)
{
    int tmp = 0;
    for (index = 0; index < size; index++)
    {
        tmp +=a;
    }
    return tmp;
}
int x[10];
...
int result =  sum(?,?,10);              //  (*)   
\end{verbatim}
}
 Assume the integer array \texttt{x[10]}, of size 10, is appropriately
initialized. Demonstrate the use of \textit{call-by-name} in the invocation of
\texttt{sum} (see the line marked (*)) by filling in the
proper actual arguments so that the call to \texttt{sum} computes
the sum of all elements from \texttt{x[0]} to \texttt{x[9]}. 
Introduce auxiliary variables where necessary. (\textbf{2p.})
\\
\end{enumerate} 

\item (Immutability.) \hfill{\textbf{[4 points]}}
\\
The \texttt{String} concatentation operator \texttt{+} in the Java
language is specified as follows:


\begin{quote}
(JLS, 15.18.1) String Concatenation Operator +

``If only one operand expression is of type String, then string conversion is performed on the other operand to produce a string at run time. The result is a reference to a String object (newly created) that is the concatenation of the two operand strings. The characters of the left-hand operand precede the characters of the right-hand operand in the newly created string. If an operand of type String is null, then the string ``null" is used instead of that operand.''
\end{quote}

 Given the following class 
in the Java language: 
{\small 
\begin{verbatim}
class GoodMorning 
{
   public static void appendMorning (String t)
   {
       t = t + "Morning";
   }
   public static void main(String [] args)
   {
       String s = "Good ";
       appendMorning(s);
       System.out.println(s);
   }
}
\end{verbatim}
}
\begin{enumerate}
The specification of the string concatentation operator allows 
the Java String class to be \textit{immutable}. Read the
specification above carefully and determine \textit{where} in the specification
immutability comes into play. Give the  phrase(s) literally, as used in 
the wording of the specification. (\textbf{2p.})
\item What does the program print? 
Draw a snapshot of the stack of activation records (``run-time stack'')
when the string concatenation operator \texttt{+} is invoked and the parameter is passed;
draw another snapshot directly after \texttt{+} has returned. 
Use the snapshots to explain. (\textbf{2p.})
\end{enumerate}


\item (Scoping.)
\hfill{\textbf{[4 points]}}
\\
Assume a for-statement in a Java-like language, that is, of the form
 
\begin{center}
  \textbf{for} (ForInit; Expression; ForUpdate) Statement.
\end{center}
Further assume the scope of a local variable declared in the 
\textit{ForInit} part of a 
for-statement includes all of the following:
\begin{itemize}
    \item Its own initializer.
    \item Any further declarators to the right in the \textit{ForInit} 
          part of the for-statement.
    \item The \textit{Expression} and \textit{ForUpdate} parts of the 
          for-statement.
    \item The contained \textit{Statement}. 
\end{itemize}

Your task:
\begin{enumerate}
\item There are other scopes besides for-loops. Give the names of two
additional examples of scopes. 
(\textbf{1p.})
\item Consider the following three snippets of for-loops. For each snippet, 
determine whether it compiles or fails with a scoping-related error
(make no additional assumptions about the variable \texttt{i}). In case
of an error, mark its position in the snippet and give a useful error message
that a compiler could provide. (\textbf{3p.})
{\small
\begin{verbatim}
//  Snippet 1
for (int i = 0; i < 10; i++) {
      ... 
}
i = 5;      

//  Snippet 2 
int i = 5;
for (int i = 0; i < 10; i++) {   
    ...
}
    
// Snippet 3
for (int i = 0; i < 10; i++) {
    ...
}

for (int i = 0; i < 10; i++) { 
    ...
}
    
\end{verbatim}
}


\end{enumerate}
\end{enumerate}

\newpage
\section{Object-Oriented Programming [12 points]}

\begin{enumerate}
\item (Smalltalk.) 
\hfill{\textbf{[4 points]}}
\\
Consider the message \texttt{lineCount} in the Smalltalk class \texttt{String}.

{\small 
\begin{verbatim}
lineCount 
"Answer the number of lines represented by the receiver, where 
every cr adds one line." 
| cr count |                                               "(1)"
cr := Character cr.                                        "(2)"
count := 1 min: self size.                                 "(3)"
self do: [:c | c == cr ifTrue: [count := count + 1]].      "(4)"
^ count                                                    "(5)"

\end{verbatim}
}
\begin{enumerate}
\item What do the bars in line (1) and the period (in lines 2-4) 
mean? Give the technical term for each. (\textbf{1p}.)
\item In lines (3) and (4): determine all messages, and 
for each message determine the receiver object and all 
argument object(s). For readability,
provide your answers in tabular form
\begin{center}
\begin{tabular}{lll}
Message & Receiver & Argument(s) \\ \hline
...     & ...      & ...     \\
\end{tabular}
\end{center}   (\textbf{3p}.)
\end{enumerate} 

\item (Algebraic specification.)  \hfill{\textbf{[4 points]}}
\\
Assume an abstract data type Natural with 4 
operations: zero,
succ, add, and is-zero. Consider zero and succ as constructors.
%is-zero as observer.
\begin{enumerate}
\item Provide an appropriate signature. (\textbf{2p}.)
\item Provide the appropriate axioms that capture the intuitive 
understanding of zero, succ, addition, and is-zero for natural numbers.
Your axioms should allow one to show the identity

\begin{center}
{add(succ(succ(zero)), succ(zero)) = succ(succ(succ(zero)))}.
\end{center} 
(\textbf{2p.})
\end{enumerate}
\newpage
\item (Polymorphic methods, static and dynamic binding.)
\hfill{\textbf{[4 points]}}
\\
Many object-oriented languages provide keywords with which  
users can control whether a method gets bound dynamically or
statically; in the pseudo language used below, \texttt{virtual}
indicates dynamic binding. 

{\small 
\begin{verbatim}
class A {
     public non-virtual void f() { print( "A.f "); }
     public virtual void g()     { print( "A.g "); }
     public non-virtual void h() { f(); g(); }
    
}
class B inherits A {
     public non-virtual void f() { print( "B.f "); }
     public virtual void g()     { print( "B.g "); }
}
int main() {
      A a;
      B b;
      // initialization of a,b
      a.h();
      b.h();
      a = b;
      a.h(); 
}
\end{verbatim}
}
\begin{enumerate}
\item What gets printed if a,b have value types?  (\textbf{2p}.)
\item What gets printed if a,b have reference types?  (\textbf{2p}.)
\end{enumerate}
\end{enumerate}

\newpage
\section*{Answers}
\section*{Imperative Programming}
\begin{enumerate}

\item Parameter passing
\begin{enumerate}
  \item Call-by-reference/value/result/value-result:
  \begin{enumerate}
    \item by-reference: 3,1
     \item by-value: 1,1
     \item by-result: garbage, garbage
     \item by-value-result: 2,1 if the return address is calculated early;
          1,2 if it is calculated late (upon return); 
          2,2 if it is calculated for x,y separately, once before and once
          after ``i++'''
   \end{enumerate}
  \item Call-by-name:
     \begin{verbatim}
      int x[10];
      int i;     // additional variable
      int result = sum(x[i],i,10);
      \end{verbatim}
\end{enumerate}
\item Immutability
\begin{enumerate}
  \item It's one phrase, used twice: ``newly created'' (the immutable
        string cannot be update in-place). 
   \item It prints ``Good\_''
         Activation records: when ``+'' is invoked, s and t both reference 
         the constant string ``Good\_'' When ``+'' returns, $t$ no longer
         aliases $s$ but refers to the anonymous object ``t + 'Morning' ''          
   \end{enumerate}


\item Scoping
{\small
\begin{verbatim}
//  Snippet 1
for (int i = 0; i < 10; i++) {
    ...
}
i = 5;            // Error, "i out of scope" or "i used but not defined" 

//  Snippet 2 
int i = 5;
for (int i = 0; i < 10; i++) {  // Error,  "i redeclared"
    ...
}
    
// Snippet 3
for (int i = 0; i < 10; i++) {
    ...
}
for (int i = 0; i < 10; i++) {   // OK: the previous i is out of scope
    ...
}
\end{verbatim}
}
\end{enumerate}
\section*{Object-Oriented Programming}
\begin{enumerate}
\item Smalltalk
% SYB, p. 57
\begin{enumerate}
\item The bar indicates the declaration of local variables, the period the
concatenation of expressions.
\item 

\begin{tabular}{lll}
Message & Receiver & Arguments \\ \hline
size    & self     & none \\
min:    & 1        & (self size) \\
do:     & self     & outer block \\
==      & c        & cr \\
ifTrue: & (c == cr) & ([count := count + 1]) \\
+       & count     & 1 \\
\end{tabular}
\end{enumerate}
\item Algebraic specification
\begin{verbatim}
sort Nat imports boolean
operations:
  zero: Nat
  succ: Nat -> Nat
  add: Nat x Nat -> Nat
  is-zero: Nat -> boolean
variables:
  n, m: Nat
axioms:
  add(zero,n)    = n
  add(succ(n),m) = succ(add(n,m))

  is-zero(zero)    = true
  is-zero(succ(n)) = false
\end{verbatim}
\item Polymorphic methods
\begin{enumerate}
\item Value variables
{\small
\begin{verbatim}
A.f
A.g
A.f
B.g
A.f
A.g
\end{verbatim}
}
\item Reference variables (``polymorphic'' variables)
{\small
\begin{verbatim}
A.f
A.g
A.f
B.g
A.f
B.g
\end{verbatim}
}
\end{enumerate} 
\end{enumerate}
\end{document}

