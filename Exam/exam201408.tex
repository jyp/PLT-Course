\documentclass{article}
\usepackage{graphicx}
\usepackage{amsmath}
\usepackage[utf8]{inputenc}
\usepackage{minted}

\setlength{\parindent}{0pt}
\setlength{\parskip}{6pt plus 2pt minus 1pt}

\usepackage{pgf, tikz}
\usetikzlibrary{shapes}
\usetikzlibrary{shapes.multipart}
\usetikzlibrary{arrows}


\newcommand{\Blue}[1]{\color{blue}#1\color{black}\xspace}

\usepackage{array}
% This is needed because raggedright in table elements redefines \\:
\newcommand{\PreserveBackslash}[1]{\let\temp=\\#1\let\\=\temp}
\let\PBS=\PreserveBackslash

%%%%%%%%%%%%%%
\newcommand{\solution}[1] {}
%\newcommand{\solution}[1] {\textbf{Solutions:}\\ #1}

%\newcommand{\comment}[1]{}
\newcommand{\answer}[0]{\paragraph{Answer:}}
% \newcommand{\answer}[1]{}
\newcommand{\comment}[1]{\marginpar{#1}}


\begin{document}

\setlength{\parskip}{2pt}

\newcommand{\examtime}{Afternoon, August 29th, 2014  (Location:V)}
\newcommand{\points}[1]{\marginpar{\bf #1 points}}
\noindent
\begin{tabular}{lr}
CHALMERS TEKNISKA H\"OGSKOLA &\examtime{}.\\
Dept. of Computer Science and Engineering & Programming Paradigms\\
Jean-Philippe Bernardy                 & DAT121 / DIT331(GU) \\
\end{tabular}

\vspace{2.5cm} \noindent
\begin{center} {\LARGE
Exam in Programming Paradigms}
\end{center}

\vspace{1.5cm}

\noindent
\examtime{}.\\
\begin{tabular}{ll}
\textbf{Examiner:} & Jean-Philippe Bernardy \\
\textbf{Permitted aids:} & Pen and blank paper.
\end{tabular}
\vspace{1cm}

\noindent
% English-Swedish or English-other language dictionary.

There are 5 questions, each worth 12 points. The total sum is 60
points.

Some questions come with \emph{remarks}: you must take those
into account.
Some questions come with \emph{hints}: even though they are meant to help you, you may ignore those.

You will be asked to write programs in various paradigms. Choose the
language appropriately in each case, and indicate which you choose at
the beginning of your answer.

\begin{tabular}[p]{cc}
  Paradigm & Acceptable language \\ \hline
  Imperative   & C or (an OO language where you refrain to use Objects) \\
  Object oriented & C++ or Java \\
  Functional & Haskell, ML \\
  Concurrent & Erlang, Concurrent-Haskell \\
  Logic & Prolog, Curry
\end{tabular}

You may also use pseudo-code resembling an actual language in the
relevant list. In that case, make sure your code can only be
interpreted in the way you intent (the responsibility lies with
you). In particular, in the case of functional/logic languages,
omitting necessary parentheses is NOT acceptable: \texttt{a b c} means
\texttt{(a b) c} and is not acceptable pseudo-code for \texttt{a (b
  c)}.


\textbf{Chalmers:}
24 points is required to pass (grade 3), 36 points is required for
grade 4, and 48 points is required for grade 5.

\textbf{GU:}
24 points is required to pass (grade G) and 42 points is
required for grade VG.

\section{Recursion and Loops}

Consider the following C code:
\begin{minted}{c}
  int s(int n) {
    int t = 0;
    while (n != 1) {
      t = t+1;
      if (n & 1 == 0)
        n = n/2;
      else
        n = 3*n + 1;
    }
    return t;
  }
\end{minted}

Convert the above to a functional program, which does not use state (modifiable variables).

Remark: you should use recursion to do so.

Hint: the operator \texttt{\&} translates to \texttt{.\&.} in Haskell.

\answer{
  \begin{minted}{haskell}
s 1 t = t
s n t = if n .&. 1
          then s (n/2)   (t+1)
          else s (3*n+1) (t+1)
  \end{minted}
}

\section{Algebraic Types and Objects}

Consider the following Haskell code:
\begin{verbatim}
data Expr = Const Int | Plus Expr Expr | Mult Expr Expr

eval (Const x) = x
eval (Plus x y) = eval x + eval y
eval (Mult x y) = eval x * eval y
\end{verbatim}

Translate the code to an object oriented language. 

\begin{itemize}
\item Write an interface corresponding to the declaration of Expr. It should contain a method corresponding to \texttt{eval} \points{4}
\item Write a class for for the \texttt{Const} case. \points{4}
\item Write a class for for the \texttt{Plus} case. \points{4}
\end{itemize}

Hint: you may leave the \texttt{Mult} case out, as it is the same as \texttt{Plus}.
\answer{
  \begin{minted}{java}
    interface Expr { int eval(); }

    class Const implements Expr { int value; int eval() {return
        value;} }

    class Plus implements Expr { Expr l; Expr r; int eval() {return
        r.eval() + r.eval();} 
    }
\end{minted}
}
\newpage
\section{Pattern matching and Higher-Order Abstractions}

Consider the following code:
\begin{minted}{haskell}

foobar :: [Int] -> Int
foobar []     = 0
foobar (x:xs)
  | x > 20     = (5*x-20) + foobar xs
  | otherwise = foobar xs

sum [] = 0
sum (x:xs) = x + sum xs

map f [] = []
map f (x:xs) = (f x) : (map f xs)

filter f [] = []
filter f (x:xs) = if f x then x:(filter f xs) else filter f xs

fold k f [] = k
fold k f (x:xs) = f x (fold k f xs)
\end{minted}

\begin{itemize}
  \item Express the function \texttt{foobar} in terms of \texttt{sum}, \texttt{map} and \texttt{filter}.\points {6}
  \item Express the function \texttt{foobar} in terms of \texttt{fold} ONLY.\points {6}
\end{itemize}

Remark: In particular you may NOT use recursion (directly) in your definitions.

\answer{
  \begin{minted}{haskell}
  foobar = sum . map (\x -> 7*x+2) . filter (>3)
  foobar = fold 0 f where
     f x acc = if x > 3 then 7*x + 2 + acc else acc
   \end{minted}
}

\newpage
\section{Closures}

Consider the following piece of Haskell code.
\begin{minted}{haskell}
data Point = Point Float Float
type Region = Point -> Bool

(.+.) :: Point -> Point -> Point
Point x1 y1 .+. Point x2 y2 = Point (x1 + x2) (y1 + y2)

opposite :: Point -> Point
opposite (Point x y) = Point (negate x) (negate y)

(.-.) :: Point -> Point -> Point
p1 .-. p2 = p1 .+. opposite p2

norm2 :: Point -> Float
norm2 (Point x y) = x*x + y*y

outside :: Region -> Region
outside r = \p -> not (r p)

intersect :: Region -> Region -> Region
intersect r1 r2 p = r1 p && r2 p

withinRange :: Float -> Point -> Region
withinRange range p1 p2 = norm2 (p1 .-. p2) <= range * range
\end{minted}

Transform the above code to use explicit closures for the
\texttt{Region} type.

\begin{itemize}
\item Create a data type containing a constructor for each operation constructing a \texttt{Region} \points{3}
\item Give the type of a function \texttt{apply} which converts each constructor of the \texttt{Region} data type into a function (\texttt{Point -> Bool}). \points{3}
\item Give its definition \points{3}
\item Give the new code for each operation, by referencing to the constructors of the new \texttt{Region} data type \points{3}
\end{itemize}

\answer{
  \begin{minted}{haskell}
data Region = Outside Region
            | Intersect Region Region
            | Within Float Point

apply :: Region -> Point -> Bool
apply (Outside r) p       = not (apply r p)
apply (Intersect r1 r2) p = r1 p && r2 p
apply (Within range p1) p2 = norm2 (p1 .-. p2) <= range * range

outside :: Region -> Region
outside r = Outside r

intersect :: Region -> Region -> Region
intersect r1 r2 = Intersect r1 r2

withinRange :: Float -> Point -> Region
withinRange range p1 p2 = Within range p1 p2
  \end{minted}
}

\newpage
\section{Functions to Relations}

Consider the following Haskell code:
\begin{minted}{haskell}
data Nat = Zero | Suc Nat

fib :: Nat -> Nat
fib Zero = Zero
fib (Suc Zero) = (Suc Zero)
fib (Suc (Suc n)) = fib (Suc n) + fib n

(+) :: Nat -> Nat -> Nat
Zero + n = n
Suc n + m = Suc (n + m)
\end{minted}

Translate the function \texttt{fib} to a relation \texttt{fibo}, such that
\begin{verbatim}
       fibo x y      is equivalent to      fib x == y
       plus x y z    is equivalent to      x + y == z
\end{verbatim}

\begin{enumerate}
\item Write the type of the relations \texttt{fibo} and
  \texttt{plus}. \points 6
\item Write their code. \points 6
\end{enumerate}

Remark: you cannot use any other helper function in your answer, only
constructors and relations. The types should be written in Curry or
Haskell-like syntax.

\answer{
\begin{minted}{haskell}
    fibo :: Nat -> Nat -> Success
    fibo Zero Zero = success
    fibo (Suc Zero) (Suc Zero) = success
    fibo (Suc (Suc n)) m = fibo (Suc n) a &
                           fibo n b &
                           plus a b m
      where a b free

    plus :: Nat -> Nat -> Nat -> Success
    plus Zero m n = m =:= n
    plus (Suc m) n (Suc p) = plus m n p
  \end{minted}
}
\end{document}
