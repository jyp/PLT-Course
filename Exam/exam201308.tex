\documentclass{article}
\usepackage{graphicx}
\usepackage{amsmath}
\usepackage[utf8]{inputenc}
\usepackage{minted}

\setlength{\parindent}{0pt}
\setlength{\parskip}{6pt plus 2pt minus 1pt}

\usepackage{pgf, tikz}
\usetikzlibrary{shapes}
\usetikzlibrary{shapes.multipart}
\usetikzlibrary{arrows}


\newcommand{\Blue}[1]{\color{blue}#1\color{black}\xspace}

\usepackage{array}
% This is needed because raggedright in table elements redefines \\:
\newcommand{\PreserveBackslash}[1]{\let\temp=\\#1\let\\=\temp}
\let\PBS=\PreserveBackslash

%%%%%%%%%%%%%%
\newcommand{\solution}[1] {}
%\newcommand{\solution}[1] {\textbf{Solutions:}\\ #1}

%\newcommand{\comment}[1]{}
% \newcommand{\answer}{\paragraph{Answer:}}
\newcommand{\answer}[1]{}
\newcommand{\comment}[1]{\marginpar{#1}}


\begin{document}

\setlength{\parskip}{2pt}

\newcommand{\examtime}{14:00, August 30th, 2013}
\newcommand{\points}[1]{\marginpar{\bf #1 points}}
\noindent
\begin{tabular}{lr}
CHALMERS TEKNISKA H\"OGSKOLA &\examtime{}.\\
Dept. of Computer Science and Engineering & Programming Paradigms\\
Jean-Philippe Bernardy                 & DAT121 / DIT331(GU) \\
\end{tabular}

\vspace{2.5cm} \noindent
\begin{center} {\LARGE
Exam in Programming Paradigms}
\end{center}

\vspace{1.5cm}

\noindent
\begin{tabular}{ll}
\textbf{Examiner:} & Jean-Philippe Bernardy
\end{tabular}
\vspace{1cm}

\noindent
Permitted aids: Pen and paper.
% English-Swedish or English-other language dictionary.

There are 6 questions: each worth 10 points. The total sum is 60
points.

Some questions come with \emph{remarks}: you must take those
into account.  
Some questions come with \emph{hints}: you may ignore those.

You will be asked to write programs in various paradigms. Choose the
language appropriately in each case, and indicate which you choose at
the beginning of your answer.

\begin{tabular}[p]{cc}
  Paradigm & Acceptable language \\ \hline
  Imperative   & C or (an OO language where you refrain to use Objects) \\
  Object oriented & C++ or Java \\
  Functional & Haskell, ML \\
  Concurrent & Erlang, Concurrent-Haskell \\
  Logic & Prolog, Curry
\end{tabular}

You may also use pseudo-code resembling an actual language in the
relevant list. In that case, make sure your code can only be
interpreted in the way you intent (the responsibility lies with
you). In particular, in the case of functional/logic languages,
omitting parentheses is NOT acceptable: \texttt{a b c} is not
acceptable pseudo-code for \texttt{a (b c)}.


\textbf{Chalmers:}
24 points is required to pass (grade 3), 36 points is required for
grade 4, and 48 points is required for grade 5.

\textbf{GU:}
24 points is required to pass (grade G) and 42 points is
required for grade VG.

\section{Objects as records}

Consider the following class hierarchy written in a Java-like
language:
\begin{minted}{java}
class A {
  float x;
  float y;
};

class B {
  int n;
  void f(A p) {
    <BODY 1>
  };

};

class C extends B {
  int m;
  // the following method overrides f from class B
  void f(A p) { 
    <BODY 2>
  }
};
\end{minted}

Translate the above classes to records with explicit method pointers. \points{4}

Complete your solution by drawing a diagram of a memory area
corresponding to an object of type each type \texttt A,\texttt B,\texttt C.
\begin{itemize}
\item \texttt A. \points{1}
\item \texttt B. \points{2}
\item \texttt C. \points{3}
\end{itemize}
Remarks: make sure all the fields are initialized to some value. The
values can be arbitrary but must be consistent with the idea of the
translation to records with explicit method pointers. If your solution
contains pointers, you must show what they point to.

\answer{
\begin{minted}{c}
    struct A {
      float x = 0
      float y = 0
    }
    struct B {
      int n = 0
      void* f(B* this, A* p) = <BODY1>;
    }
    struct C {
      int n = 0
      void* f(C* this, A* p) = <BODY2>;
      int m;
    }
\end{minted}
}


\newpage
\section{Calls and recursion}
Consider the function
\begin{verbatim}
procedure fib(n : Int)
  if n == 0
    return 0
  else if n == 1
    return 1
  else 
    tmp1 := fib (n-1)
    tmp2 := fib (n-2)
    return tmp1 + tmp2
\end{verbatim}

Remove recursion, using an explicit stack. \points {10}

Remarks

\begin{itemize}
\item  You should assume a ``global'' stack accessed via \texttt{push},
  \texttt{pop} and \texttt{top} primitives.

\item You must implement \emph{the same algorithm}. Do not change the
  algorithm in the process of doing the translation. Do not apply
  tail-call optimisation.
\end{itemize}

Hints: you should first define the type of values that you can push on
the stack, and you can assume that labels can be used as values.

\answer{
  \begin{minted}{c}
    
Stack frame: (n:Int,tmp:Int,caller:Pointer)

fib:
  if top.n == 0
     result = 0;
     goto top.caller
  else if top.n == 1
     result = 1
     goto top.caller
  else
     push (n-1,X,ret1)
     goto fib
ret1:
     pop;
     top.tmp = result;
     push (n-2,X,ret2);
     goto fib;
ret2:
     pop;
     result := result + top.tmp;
     goto top.caller;  
  \end{minted}
}


\newpage
\section{Pattern matching and Higher-Order Abstractions}

Assume the following Haskell data type:
\begin{minted}{haskell}
data List = Nil | Cons Int List
\end{minted}
and consider the following code:
\begin{minted}{haskell}

foobar :: List -> Int
foobar []     = 0
foobar (x:xs)
  | x > 3     = (7*x + 2) + foobar xs
  | otherwise = foobar xs

sum [] = 0
sum (x:xs) = x + sum xs

map f Nil = Nil
map f (Cons x xs) = Cons (f x) (map f xs)

filter f Nil = Nil
filter f (Cons x xs) = if f x then Cons x (filter f xs) else filter f xs

fold k f Nil = k
fold k f (Cons x xs) = f x (fold k f xs)
\end{minted}

Express the function \texttt{foobar} in terms of \texttt{sum}, \texttt{map} and \texttt{filter}.\points {5}
Express the function \texttt{foobar} in terms of \texttt{fold} ONLY.\points {5}

Remark: In particular you may NOT use recursion (directly) in your definitions.

\answer{
  \begin{minted}{haskell}
  foobar = sum . map (\x -> 7*x+2) . filter (>3)
  foobar = fold 0 f where
     f x acc = if x > 3 then 7*x + 2 + acc else acc    
   \end{minted}
}


\newpage
\section{Closures}

You will demonstrate how to transform higher order functions to first order ones,
by using explicit closures.  That is, use a Haskell \emph{data
  structure} to represent $\lambda$-expressions.

Consider the function filter:
\begin{minted}{haskell}
filter f Nil = Nil
filter f (Cons x xs) = if f x then Cons x (filter f xs) else filter f xs
\end{minted}

\begin{itemize}
\item Translate the code of \texttt{filter}. \points{2}
\item Give the type of the closure application function used in the
  translation. \points{2}
\end{itemize}

Consider the following use of \texttt{filter}:
\begin{minted}{haskell}
above x xs = filter (greaterThan x) xs
greaterThan x y = x < y
\end{minted}

\begin{itemize}
\item Give the representation \texttt{greaterThan} in the translation. \points{2}
\item Translate the function \texttt{above} to use explicit closures. \points{2}
\item Give the code of the closure application function. \points{2}
\end{itemize}

\answer{
  \begin{minted}{haskell}
    
1.

  filter f Nil = Nil
  filter f (Cons x xs) = if f `ap` x then Cons x (filter f xs) else filter f xs

2. ap :: Fun -> Int -> Bool

3. 

data Fun = GreaterThan Int 

4. above x y = filter (GreaterThan x) xs

5. ap (GreaterThan x) y = x < y
 
  \end{minted}
}

\section{Continuations}

Consider the following function, which solves the Hanoi tower problem:
\begin{verbatim}
hanoi 0 s i d = return ()
hanoi n s i d = do
    hanoi (n-1) s d i
    move s d 
    hanoi (n-1) i s d 
\end{verbatim}

Transform \texttt{hanoi} to use explicit continuations. \points{10}

Remarks: 
\begin{itemize}
\item In the translation, you will use a different version of
  \texttt{move}, which takes an explicit continuation as argument.
\item The translation of \texttt{hanoi} should also take an explicit
  continuation as argument.
\item Do not change the algorithm. Do not ``optimize'' it. The call
  structure should remain the same.
\end{itemize}

\answer{
hanoi 0 s i d k = k ()
hanoi n s i d k = 
    hanoi (n-1) s d i (
    move s d          (
    hanoi (n-1) i s d 
    k))
}

\section{Relations}

Consider the following Haskell code:

\begin{minted}{haskell}
data List = Nil | Cons Int List
data Tree = Tip | Bin Tree Int Tree

append :: List -> List -> List
append Nil xs = xs
append (Cons x xs) ys = Cons x (append xs ys)

flatten :: Tree -> List
flatten Tip = Nil
flatten (Bin a x b) = append (flatten a) (Cons x (flatten b))
\end{minted}

Translate the above functions to relations \texttt{flatten'} and
\texttt{append'} such that:
\begin{minted}{haskell}
       append' x y z       is equivalent to      append x y == z
       flatten' x y        is equivalent to      flatten x == y
\end{minted}

\begin{enumerate}
\item Write the type of the relations \texttt{flatten'} and
  \texttt{append'}. \points 4
\item Write their code. \points 6
\end{enumerate}

Remark: you cannot use any other helper function in your answer, only
constructors and relations. The types should be written in Curry or
Haskell-like syntax.

You may write the code in Curry syntax or Prolog syntax. 
\answer{
\begin{minted}{haskell}
    append :: List -> List -> List -> Success
    append []     ys zs = ys =:= zs
    append (x:xs) ys zs = append xs ys zs' `and`
                          zs =:= x:zs'
       where zs' free

   flatten :: Tree -> List -> Success
   flatten Tip Nil = success
   flatten (Bin a x b) zs = flatten a xs `and` flatten b ys `and` append xs (Cons x ys) zs
      where xs, ys free
\end{minted}
}

\end{document}
