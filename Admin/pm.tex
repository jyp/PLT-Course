\documentclass[oneside,11pt]{article}
\usepackage[paper=a4paper,margin=25mm,bottom=40mm]{geometry}
\usepackage{times}
\usepackage{fancyhdr}
\usepackage{graphicx}
\pagestyle{fancy}
\setlength{\parindent}{0mm}
\rhead{}
\chead{}
\lhead{\includegraphics{ChalmGUtextsvEng.pdf}}
\lfoot{
\sf \small
\begin{tabular}[b]{lr}
CHALMERS TEKNISKA HÖGSKOLA AB\\
GÖTEBORGS UNIVERSITET\\
\textit{Data- och informationsteknik}\\
412 96 Göteborg\\
Telefon: 031-772 10 00 (CTH) / 031-786 00 00 (GU)
\end{tabular}
}
\cfoot{}
\rfoot{\includegraphics{ChalmGUmarke.pdf}}


\begin{document}

{
\vspace{1cm}
\huge
DAT121/DIT331 Programming Paradigms VT 2014
\vspace{1cm}
}

{\bf
\begin{tabular}[t]{ll}
Examiner and Course-Responsible: & Jean-Philippe Bernardy\\
Course Assistants: & Nikita Frolov, Dan Rosén
\end{tabular}
}
\hrule

\section*{Course content}

This course provides an overview of common programming paradigms,
including imperative, object-oriented, logic, concurrent, and
functional programming, and discusses the fundamental concepts
underlying the design, definition, and implementation of modern
computer languages. Students get practical experience with languages
that exemplify a particular paradigm.

The course contains:
\begin{itemize}
\item an introduction by example to each paradigm;
\item theoretical and practical studies of transformations between
  paradigms
\end{itemize}

Examples of such transformations include:
\begin{itemize}
  \item introduction/removal of jumps/loops,
  \item introduction/removal of recursion/stack,
  \item introduction/removal of higher order functions/closures
  \item introduction/removal of processes/continuations,
  \item introduction/removal of explicit search.
\end{itemize}

\section*{Learning outcomes}

\begin{description}

\item[Knowledge and understanding]
    After completion of the course the student is expected to be able to
  \begin{itemize}
  \item Explain and contrast the principles of different paradigms both conceptually and in terms of particular language features.
  \item Know the relationship between mainstream programming languages, the features they implement, and the paradigms they support.
  \end{itemize}

\item[Skills and abilities]
    After completion of the course the student is expected to be able to
  \begin{itemize}
  \item Write small idiomatic programs in languages that represent different paradigms.
  \item Read programs written idiomatically in a given paradigm, and translate (encode) them into a language that does not support the paradigm directly.
  \item Read non-idiomatic programs (that use an encoding), and re-write them in their idiomatic paradigm.
  \end{itemize}

\item[Judgement and approach]
    After completion of the course the student is expected to be able to
  \begin{itemize}
  \item Evaluate and apply the styles and strategies that characterize different paradigm and assess their suitability for solving a given problem.
  \item Recognize the paradigms at the core of programs, regardless of shallow/accidental implementation choices. 
  \end{itemize}
\end{description}

\section*{Course structure}
The course consists of a lectures introducing the contents, and
exercise sessions where students consolidate their understanding and
skills.
\section*{Examinations forms}
The course is examined by an individual written examination, carried
out in examination hall 8:30, March 15th, 2014

There are 5 questions, each worth 12 points. The total sum is 60
points.

\textbf{Chalmers:}
24 points is required to pass (grade 3), 36 points is required for
grade 4, and 48 points is required for grade 5.

\textbf{GU:}
24 points is required to pass (grade G) and 42 points is
required for grade VG.

Permitted aids: pen and blank paper.

\section*{Course literature}
Lecture notes on the web page.
\section*{Schedule}

\begin{center}
\begin{tabular}{rrllll}
Week & Start date & Lecture 1 & Lecture 2 & Exercises & TAs\\
\hline
1 & 0120 & Intr & IP 1 & EX 1 & Ina + Nik\\
2 & 0127 & IP 2 & OO & EX 2 & Nik + Ina\\
3 & 0203 & FP 1 & No lecture ($\ast$) & No exercise ($\ast$) & \\
4 & 0210 & FP 2 & FP 3 & EX 3 & Ina + Dan\\
5 & 0217 & CP 1 & CP 2 & EX 4 & Nik + Dan\\
6 & 0224 & LP 1 & LP 2 & EX 5 & Dan + Ina\\
7 & 0303 & Review & Guest Lecture? & EX 6 & Dan + Nik\\
\end{tabular}
\end{center}

($\ast$): Charm Days

See table below for detail of contents.

\begin{center}
\begin{tabular}{ll}
Abbr & Contents\\
\hline
Intr & Models of computations; types; abstraction\\
IP 1 & Goto ↔ Loops, Inlining procedures, Procedures → Gotos, Pointers and passing by reference\\
IP 2 & Recursion. Substitution model in the presence of recursion, Explicit stack\\
OO & Inheritance, Interfaces, Notion of co/contra variance\\
FP 1 & Algebraic Types (Pattern matching, Parametric types, Currification)\\
FP 2 & HO Abstraction, Polymorphic functions, Translating away HOF (Inlining, Closures)\\
FP 3 & Explicit state, purity and laziness (+ continuations?)\\
CP 1 & Resource-managing processes\\
CP 2 & Explicit continuations\\
LP 1 & Intro to LP; Unification\\
LP 2 & Functions to Relations, Search as list of successes\\
Review & Solutions to (part of) sample exams\\
\end{tabular}
\end{center}

%% \section*{Additional information}


\end{document}
