\documentclass{article}
\usepackage{fancyvrb,hyperref,listings}

\newcommand{\Cpp}{C\kern-0.05em\texttt{+\kern-0.03em+}}
\newcommand{\cpp}{\Cpp}
\newcommand{\mycomment}[1]{}

\lstset{language=java,basicstyle=\small}
\lstset{commentstyle=\textit,flexiblecolumns=true,mathescape=true}
\lstnewenvironment{java}
    {\lstset{language=java,frame=tb}}
    {}
\lstnewenvironment{java2}
    {\lstset{language=java}}
    {}
\lstnewenvironment{cplus}
    {\lstset{language=c++,frame=tb}}
    {}
\lstnewenvironment{cplus2}
    {\lstset{language=c++}}
    {}
\lstnewenvironment{perl}
    {\lstset{language=perl,mathescape=false,lineskip=0\baselineskip,frame=tb}}
    {}

\newcommand{\percent}[1]{\protect \marginpar[l]{\bf [#1 points]}}
\newcounter{question}
\newenvironment{question}[1]{
  \addtocounter{question}{1}
  \paragraph{Exercise~\arabic{question}  \percent{#1}}
}{
  \vfil
}

\begin{document}

\title{Exercises on Imperative Programming\\
  \large Programming Paradigms}
\author{Sibylle Schupp}
\maketitle


\section{Parameter passing \percent{25}}

\begin{question}{X}{
For each of the six parameter passing mechanisms discussed in class,
determine the final values of the array \textsf{A} after the call
to \textsf{f}. 

\vfil\noindent\begin{minipage}[b]{0.5\linewidth}
\begin{java}
void f(int x, int y) {
    x = 1;
    y = 3;
}
\end{java}%
\end{minipage}%
\begin{minipage}[b]{0.5\linewidth}
\begin{java}
int[] A = new int[2];
A[0] = 0;
A[1] = 2;
f(A[0],A[A[0]]);
\end{java}%
\end{minipage}%
\end{question}

\begin{question}{X}{
An important question of call-by-result, which was not discussed in
class, is the question \textit{when} the lvalue of the actual parameter is 
computed. One might expect that this happens before the callee executes,
but that is not necessarily the case---since the value of the actual
parameter is not needed until the 
callee returns, its computation could be delayed until the value is needed.
Ada, for example,  supports early lvalue-computation, while Algol-W implements
late lvalue-computation.

Use the snippet below to explain the difference between early and late
lvalue-computation.
\vfil\noindent\begin{minipage}[b]{0.45\linewidth}
\begin{java}
void f(int [] A) {
   int i = 0;
   g(i, A[i]);
   // What happens to i and A?
}

\end{java}%
\end{minipage}%
\begin{minipage}[b]{0.55\linewidth}
\begin{java}
void g(by-reference int i, by-result int j) {
   j = 2;
   i = 1;
   j = 3;
}
\end{java}%
\end{minipage}%
\end{question}

%\begin{question}{X}{
% When does C\# perform the lvalue-computation for its by-result parameters?
%\end{question}

\begin{question}{X}{
Another important detail of call-by-result and call-by-value-result concerns
the order in which the final values of the formal parameter are written back
to the actual parameters. For a binary method \textsf{f(x,y)}, for
example, either (the value of) \textsf{x} or (the value of) 
\textsf{y} is written back
first; this order might make a difference. 

Write two methods \textsf{f,g} (in pseudo-code,
if you wish) that demonstrate the difference.
\end{question}


\section{Activation records and scoping \percent{25}}
The \textit{scope} of a variable declaration refers to the range of
statements in which the variable can be referenced. 
\begin{itemize}
\item In \textit{static scoping}, the use of a variable $y$ in
a subprogram $f$ refers to the declaration of $y$ at the beginning of $f$ 
or, if not
declared there, to the declaration of $y$ in the subprogram $g$ 
whose declaration
contains the declaration of $f$, and so on. The static scope, thus,
is a property of the program text and can be determined at compile time.
\item In \textit{dynamic scoping}, on the other hand, 
the scope depends on the call sequence of
subroutines: the use of a variable refers to the declaration in 
the nearest caller (``dynamic parent'').
\end{itemize}
Java, C, \Cpp, C\#, and most other
 languages support static scoping. The language Perl is special
since it gives the user control: 
variables declared as \textsf{`my'} are statically scoped, 
variables declared as \textsf{'local'} are dynamically scoped.  

\mycomment{
In this exercise, the question is: given a particular  use of
a variable, how does the run-time system know which declaration that 
variable refers to? As discussed in class, activation records contain
only two kinds of internal information: the return address and the previous
activation record. This does not suffice for languages with
static scoping and \textit{non-local variable references}.
}

\begin{question}{X}
  What is the effect of static and dynamic scoping? Consider the Perl
  program below and vary between \textsf{my} and \textsf{local} for the
  three declarations of $x$. When, if at all, does the value $x=2$ get
  printed? When $x=4.14$ or $x=56$ (if at all)?
\end{question}

\paragraph{Hint:}
Perl is installed on (almost) all machines; you can run the program.

\begin{question}{X}
  In dynamic scoping, it is easy for the run-time system to determine
  which declaration the use of a variable refers to. Explain using the
  run-time stack.  
\end{question}

\mycomment{
Static scoping with non-local references, on the other hand,
is not so easy to implement. In particular, the kind of 
activation records discussed in class (which contain
only two kinds of internal information: the return address and the previous
activation record) does not suffice. 
%

What information would be necessary and where could it be stored? 
Make a suggestion  and 
sketch a (simple) algorithm that takes a run-time stack and an activation record,
and returns the ``right'' reference (or at least its activation record).
How expensive is your algorithm? Determine the parameters on which its
cost depends and give the cost of the algorithm.
}

\begin{perl}
 sub outer {
   my $x = 0;

   sub r(){
       $x = $x + 1; 
       print("In subroutine r, x = ", $x);
   }
   sub q() {
      my $x = 3.14; 
      &r;      # call outer's subroutine r
   }
   sub p() {
      my $x = 55; 
      &q;      # call outer's subroutine q
   } 
   &p;         # call subroutine p
}
&outer;
\end{perl}
\newpage 
\section{Duff's device [25p]}
Duff's device is an optimization idiom for serial copies 
in the language C. The fragment below lists first the original code,
then Duff's optimization.
\begin{cplus}
/* (Almost) original code */
int main() {
   register short *to, *from;
   register int count;	
   ... 
   {
     /* pre: count > 0 */		
     do
	  *to++ = *from++;		
     while(--count>0);	
   }
   return 0; 
}

\end{cplus}

\begin{cplus}
/* Duff's transformation */
int main() {
  register short *to, *from;
  register int count;	
  ...
  {
     /* pre: count > 0 */		
     int n = (count + 7) / 8;
     switch(count % 8){
       case 0:	do{	*to++ = *from++;
       case 7:	        *to++ = *from++;
       case 6:	        *to++ = *from++;
       case 5:	        *to++ = *from++;
       case 4:	        *to++ = *from++;
       case 3:	        *to++ = *from++;
       case 2:	        *to++ = *from++;
       case 1:	        *to++ = *from++;
                } while (--n > 0);
     }
  }
  return 0;
}
\end{cplus}
Your task
\begin{enumerate}
%\item Explain pointer variables in terms of r-values and l-values: what is the r-value of a pointer?
\item What happens in the assignment ``\textsf{*to++ = *from++;}''?  Explain. 
\item Look up the storage class \textsf{register}. Furthermore, 
look up what the   \textsf{switch} statement does (preferably
in the language specification) and briefly
explain the second program.
\item Why did Duff expect the second program to be faster? Speculate and
give a brief answer.
\end{enumerate}

\paragraph{Further reading}
\begin{itemize}
\item For the original email in which Tom Duff presented his ``device'',
see 

\textsf{http://www.lysator.liu.se/c/duffs-device.html}
\item The specification of the C language is available at 

\textsf{http://www.open-std.org/jtc1/sc22/wg14/www/docs/n1124.pdf}
\item Check the generated code yourself: compile it
with \textsf{gcc -S $<$filename$>$}.
\end{itemize}

\vfill

\end{document}