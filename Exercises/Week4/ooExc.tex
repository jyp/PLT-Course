\documentclass{article}

\input{header}
\begin{document}
\title{Ex. 4: Programming Paradigms 
\\
Object-Oriented Programming}
\author{\courseinfo}
\date{}
\maketitle
\mycomment{
\paragraph{Submission}
Submit your answers (one per pair of students)
electronically through the Fire system before the deadline.
If you haven't registered yet with Fire 
do so as soon as possible 
(\textsf{https://fire.cs.chalmers.se:8042/cgi/Fire-prgprd}).


%
\textit{To pass the course, you must pass 25\% of all problems on \textbf{this}
sheet and 50\% of all problems total. For some problems, you may earn
extra points.}

\begin{center}
\fbox{Deadline: Thursday, Nov. 15, 2007 XXX, midnight (Swedish time).}
\end{center}
%
The deadline is not extensible. 

\paragraph{Academic integrity} is important. If you submit solutions that
are not your own, we hand the case over to the Chalmers officials, as specified
here 
\begin{center}
\textsf{http://www.chalmers.se/en/sections/education/current\_students/}\\
\textsf{\-joint\_rules\_and\_dire/rules\_of\_discipline} (English) 
%
and \\
%
\textsf{http://www.chalmers.se/sections/ar\_student/regler\_och\_riktlinje/disciplinstadga}(Swedish).
\end{center}

\paragraph{Help} Don't hesitate to ask for help!
The course assistants are 
Gustav Munkby and Jean-Philippe Bernardy.
You can reach them at the following email addresses:
%
\begin{center} 
\textsf{munkby-at-cs-chalmers.se}  and 
\textsf{bernardy-at-chalmers.se} 
\end{center} 
%
(replace ``-'' by  ``.'', and ``at'' by the at-sign)

\newpage
}

\section{Algebraic specification [25~points]}

A standard example in algebraic specification is the data type \textit{stack}.
A textbook-like specification may support the following operations:
\begin{center}
\begin{tabular}[t]{ll}
 new & create a stack \\
 push & add an element to the top of the stack \\
 pop  & remove the top element from the stack \\
 top &  return the top element of the stack \\
 empty & return true exactly if there is no element on the stack
\end{tabular}
\end{center}


\begin{enumerate}
\item Write an algebraic specification of this stack: provide a full signature
and define its semantics through appropriate axioms. 

In your specification,
make reasonable assumptions about what a stack (informally) is; you
may ``import'' other specifications. Make explicit which operations
you consider as \textit{constructors} and informally argue why your
set of axioms is complete.
\item Now turn to the Java class \textsf{Stack}. As it is easy to see,
this class is not an implementation of your algebraic specification 
in part 1---its signature and semantics simply differs. 
Adjust your algebraic specification from 1. so that \textsf{java.util.Stack}
can be considered an implementation.

\begin{enumerate}
\item Decide how you want to handle Java exceptions.
\item Change, and extend, the signature from part 1 so that it ``fits''
the Java stack. Summarize and briefly justify all changes.
\item Provide axioms for the \textsf{search} operation. Revise the
axioms for \textsf{empty, push, peek} as necessary; briefly justify.
\item  Read the API for the \textsf{pop} method and explain the problem
for the corresponding operation in the algebraic specification 
(1-2 sentences).\\
 \textbf{Optional [3 Bonuspoints]:} provide axioms for \textsf{pop} that reflect the
Java semantics. We will accept partial answers, i.e., axioms that 
do not entirely capture that semantics, but you need to be specific
what is, and what isn't, captured. 

\item Complete the algebraic specification. 
\end{enumerate}
\end{enumerate}

Further material:
\begin{itemize}
\item The Java API:
\textsf{http://java.sun.com/j2se/1.5.0/docs/api/java/util/Stack.html}
\item Auxiliary lecture notes:
\textsf{http://www.cs.st-andrews.ac.uk/$\sim$ifs/Resources/Notes/FormalSpec/AlgebraicSpec.pdf}
\end{itemize}


\newpage
\section{Co-, Contra-, Nonvariance [25~points]}


\begin{java}
interface Monoid {
   Monoid op(Monoid second);
   Monoid id();
};

struct IntegerAdditiveMonoid : Monoid {
    public IntegerAdditiveMonoid(int x) { 
         elt = x;
    }    
    public IntegerAdditiveMonoid op(IntegerAdditiveMonoid second) { 
         return new IntegerAdditiveMonoid(
             elt + second.elt); 
    }
    public IntegerAdditiveMonoid id(){ 
         return new IntegerAdditiveMonoid(0);
    }
    int elt;
};

> gmcs Monoids.cs 
Monoids.cs(6,8): error CS0535: `IntegerAdditiveMonoid' does not implement 
  interface member `Monoid.op(Monoid)'
Monoids.cs(2,11): (Location of the symbol related to previous error)
Monoids.cs(6,8): error CS0535: `IntegerAdditiveMonoid' does not implement 
  interface member `Monoid.id()'
Monoids.cs(3,11): (Location of the symbol related to previous error)
Compilation failed: 2 error(s), 0 warnings

\end{java}
%
Your task:
\begin{enumerate}
\item Briefly recap: what is a \textit{monoid}, mathematically? Give two examples
of data types that can be considered monoids.
\item The C\#-code above does not compile. Explain the error message
in terms of co-/contra-/nonvariance.
\item What if the method \textsf{op} \textit{would} compile? 
Introduce instances \textsf{a,b} of appropriate types so that
\textsf{a.op(b)}, if compiled, would result in a run-time error.
What if the method 
\textsf{id} would compile? Could you construct a similar run-time
error? 
\mycomment{
\item Show that the C\# type system would be \textit{unsound} if 
the code would compile without error (i.e., one could get a type error at
run time). Hint: define a second implementation of the 
\textsf{Monoid} interface, with
a (necessarily) different signature for the methods \textsf{op} and 
\textsf{id}; call \textsf{IntegerAdditiveMonoid.op()} with an object that
will cause a run-time error.
}
%
\item The corresponding program in Java behaves differently. Briefly
explain how, consult the Java Language Specification (JLS), and back your answer
with the appropriate clause(s) in the JLS.
\item Suppose you changed the code so that 
the (current) C\#-compiler accepts it. What is the problem then? 
%\item How can you change the code so that it compiles? Is your workaround type-safe? 
\end{enumerate}

\paragraph{Further reading}
\begin{itemize}
\item The Mono compiler is a free compiler for C\#. Read more about it
at

\textsf{http://www.mono-project.com/CSharp\_Compiler}

\item JLS is available on-line 
 at 
\url{http://java.sun.com/docs/books/jls/third_edition/html/j3TOC.html}

\end{itemize}

\newpage
\section{Dynamic dispatching \& the visitor [25 points]}

Suppose the three classes \textsf{Shape}, \textsf{Rectangle}, and 
\textsf{Circle}, and the appropriate inheritance relation between them.
Further assume a method \textsf{intersect()} that can determine whether two
shapes overlap. Obviously, the implementation of \textsf{intersect()} 
depends on the actual shapes involved. 

\begin{enumerate}
\item Pick an object-oriented language with single-dispatch mechanism 
and design  the 3 classes so that all possible intersections
of shapes can be properly handled. 
What is the problem with single dispatching? Name at least 2 issues.
\item The following Java snippet redesigns the
class hierarchy using \textit{double dispatch}. 
Explain how double dispatching works by
giving the sequence of calls for the line marked (*) in \textsf{Main}.
Name two problems
with double dispatching.

\textbf{Optional}: the corresponding code in C\# would not 
compile. Look up the visibility rules for \textsf{protected} members
and explain which (compile-time) error one would get. 
\begin{java}
class Shape {
    public  boolean intersect(Shape s) {
         return s.intersectShape(this);
    }
    protected  boolean intersectShape(Shape s) {
        // intersect Shape x Shape
    }
    protected  boolean intersectRectangle(Rectangle r) {
        // no special code for Shape x Rectangle
        return intersectShape(r);
    }
    protected  boolean intersectCircle(Circle c) {
        return intersectShape(c);
   }
}
class Rectangle extends Shape {
    public  boolean intersect(Shape s) {
        return s.intersectRectangle(this);  
    }
    protected  boolean intersectRectangle(Rectangle r) {
        // Rectangle x Rectangle
        ..
    }
}
class Circle extends Shape {
    public  boolean intersect(Shape s) {
        return s.intersectCircle(this);  
    }
    protected  boolean intersectCircle(Circle c) {
        // Circle x Circle 
   }   
}

class Intersect {
   public static void Main() {
      Shape s1 = new Rectangle(); 
      Shape s2 = new Circle();
      boolean b = s2.intersect(s1);        // *
   }
}
\end{java}
\item The \textit{Visitor pattern} provides an alternative workaround,
also based on double dispatching. In difference
to the solution above, the pattern 
introduces two hierarchies, the visited
hierarchy and the visitor hierarchy. In the example,
 \textsf{Shape, Rectangle, Circle}
form the visited hierarchy, \textsf{IntersectVisitor} and its two
subclasses \textsf{RectangleIntersectVisitor} and \textsf{CircleIntersectVisitor}
form the visiting hierarchy. The visited classes are
augmented with a method \textsf{accept}, which uses double dispatching
to call the appropriate ShapeVisitor (and to invoke the
right computation). 
\\
Your task
\begin{itemize}
\item Complete the code for the two class hierarchies and the client.
\item Name two problems of the visitor approach.
\end{itemize}

\begin{java2}
class Shape {
    public void accept(IntersectVisitor v) {
	v.visitShape(this);
    }
}
class Rectangle extends Shape {
public  void accept(IntersectVisitor v) { 
        // your job
    }
}
class Circle extends Shape {
   // your job
}
 
// visitor hierarchy
abstract class IntersectVisitor { 
    boolean intersects; 
    public abstract void visitShape(Shape s);
    public abstract void visitRectangle(Rectangle r); 
    public abstract void visitCircle(Circle c); 
} 

// client code
  Shape s1 = new ..
  IntersectVisitor v = new RectangleIntersectVisitor();
  s1.accept(v);
  if (v.intersects) ...
\end{java2}
\end{enumerate}

\paragraph{Further reading}
\begin{itemize}
\item MultiJava is a research language that implements multi-method:\\
\textsf{http://www.cs.washington.edu/research/projects/cecil/pubs/mj-toplas.pdf}
\item An alternative to multi-methods that also avoids the problems of
single- or double dispatch, are ``open methods.'' A
 proposal exists to extend the \Cpp\ language by ``open methods'', see
document no N2216=07-0076 by the standardization committee:

\textsf{www.open-std.org/jtc1/sc22/wg21/docs/papers/2007/n2216.pdf}
\end{itemize}
\newpage
\section{A Glimpse of Squeak [25~points]}
\begin{enumerate}
\item For each of the following expressions determine
the message(s), and for each message determine the receiver object and all 
argument object(s). 
\begin{java}
Date today 
2 + 3 
anArray at: 1 put: 'hej'
10 raisedTo: 2 + 3
10 negated raisedTo: 2 + 3
2 * 3 + 4      
\end{java}
\item The following snippet contains a fragment of the class \textsf{Count} with the methods \textsf{initialize},
 \textsf{value},  \textsf{increment1}, and \textsf{increment2}:
\begin{java}
initialize
    value := 0.
value
    value notNil ifFalse: [^0].
    ^value
increment1
    value := value + 1.
increment2
    value := self value + 1
\end{java}
\begin{enumerate}
\item Compare the two increment
methods from a design point of view: which one would you prefer?  Why?
\item What is the meaning of the semicolon (``;''), thus  the value of \textsf{aCounter} at the end of  the following program (1 sentence)?

\begin{java}
|aCounter|
aCounter := Count new.
aCounter initialize.
aCounter increment1; increment2.
\end{java}
(Hint: open the ``Browser'', implement the class \textsf{Count}, enter the
 program below in 
the ``Workspace'',
run the program, and ``inspect'' the counter variable.)
\end{enumerate}
\item Consider the following block expression
\begin{java}
|anArray sum|
sum := 0
anArray := #(4 55 19 23).
anArray do: [:item | sum := sum + item].
^sum
\end{java}
\begin{enumerate}
\item Rewrite the block in Java (or a Java-like language). (This is just
a sanity check; don't make it too complicated!)
\item Rewrite the block using \textsf{inject:into}.
%\item Rewrite the block using explicit array indexing (via the method \textsf{at:}) to access array elements. 
\end{enumerate}
\end{enumerate}
Further material
\begin{itemize}
\item By default, Squeak is not installed on the machines in the department,
but you can easily install it yourself at home or in your account at
Chalmers. Visit the download area on the main webpage of Squeak:
\begin{center}
\textsf{http://www.squeak.org}
\end{center}
\item The Squeak main page contains links to tutorials of varying degree
of difficulty and lots of other useful online resources. Here is a 
tutorial for Java programmers:
\begin{center}
\textsf{http://daitanmarks.sourceforge.net/or/squeak/squeak\_tutorial.html}
\end{center}

\end{itemize}
\end{document}

\section*{Answers (sketches)}

\section*{Algebraic specification}
\begin{enumerate}
\item ADT stack 
\begin{verbatim}
sort stack imports bool, element
operations:
   new:   -> stack
   push:  element x stack -> stack
   pop:   stack -> stack
   top:   stack -> element
   empty: stack -> bool
variables:
   s: stack, e: element
axioms:
   pop(push(e,s)) = s
   top(push(e,s)) = e
   empty(new()) = true
   empty(push(e,s)) = false
\end{verbatim}
Assume that new, push are the two generators (constructors). Which axioms do you need
for pop, top, empty? 
\begin{itemize} 
\item pop: the first axiom ( pop(push(e,s)) = s ) defines
what pop does on a stack generated via push. What about the other constructor,
new? If the stack has been
created via new, there is no element that pop could remove---thus, pop
can be undefined. Since there are no generators other than push and create,
there are no further axioms for pop.
\item top: argue as with pop
\item empty: we need to define what empty does on each of the two generators.
Rest  follows directly from the
informal specification. 
\end{itemize}


\item Abstraction from the Java stack class
\begin{verbatim}
sort Stack imports Vector, E, Object, boolean, int, EmptyStackException
operations:
   Stack:   -> Stack
   push:  Stack x E -> Stack
   pop:   Stack -> E
   peek:  Stack -> E
   empty: Stack -> boolean
   search: Stack x Object -> int
variables:
   s: Stack, e: element, o: Object, error: EmptyStackException
axioms:
   pop(push(s,e)) = e
   pop(Stack) = error
   peek(push(s,e)) = e
   peek(Stack) = error
   empty(Stack) = true
   empty(push(s,e)) = false
   search(push(s,o),o) = 1
   search(push(s,e),o) = 1 + search(s,o)
   search((push(Stack,e)),o) = -1
\end{verbatim}
\begin{itemize}
\item Exceptions could be modelled just as  sorts.
\item Changes to the signature: we assume the ``this'' object as
first parameter (even though it is not explicit in Javadoc), which
means that the previous signature of ``push'' needs to swap its
two arguments. Other changes: different sort names (E instead of element, 
boolean instead of bool), different operation names.
 Big change: ``pop'' changes its return
parameter.
\item The problem of ``pop'': how to express the destructive
update? (Another problem is that, without further steps, 
``pop'' and ``top'' become indistinguishable.)
\item Changes to the axioms: we still have two generators. But
``pop'' and ``peek'' must now be defined for empty stacks, so
there is one additional axiom each (which specifies the error state). 
The axioms for ``search'' assume the semantics of ``+.''
\item How to deal with destructive updates? Eiffel's or JML's take
is to introduce a helper function (or variable) that reflects the
change to the stack, for example a ``size'' function.
\begin{verbatim}
sort Stack:
operations:
   size: Stack -> int
axioms:
   size(pop(s,e)) = size - 1
\end{verbatim}
There are two problems, though: for one, ``size'' (or related functions)
don't specify which element gets deleted; second, if they
are helper functions (or internal variables), they should not be visible.
\end{itemize}
\end{enumerate}

%  http://scom.hud.ac.uk/scomtlm/book/node221.html
%www.cs.colorado.edu/~kena/classes/5828/s00/lectures/lecture12.pdf



\section*{Squeak}

\begin{enumerate}
\item Expressions

\begin{tabular}[t]{l|lll}
           & Mesg & Receiver               & Args  \\ \hline
Date today & today & Date                  & no   \\
2 + 3      & +     & 2                        & 3   \\
anArray at: 1 put: 'hej' & at:put & anArray & 1, 'hej'  \\
10 raisedTo: 2 + 3       & +           & 2  & 3 \\
                         & raisedTo    & 10 & 5  \\
10 negated raisedTo: 2 + 3 & negated   & 10 & value of (raisedTo: 2 + 3) \\
                           & raisedTo  & -10 & value of (2+3) \\
                           & +         & 2   & 3 \\
2 * 3 + 4                  & *         & 2   & value of (3+4) \\
\end{tabular}
\item Counter: the semicolon means ``cascading'', so the value is ``2.'' 
Note the overloading of ``value'': increment1 updates the instance
variable ``value'' directly, while increment2 sends the ``value'' message
to ``self.'' Thus, increment1
establishes a direct dependence on the implementation of the class, which
could be considered a disadvantage from the maintenance point of view. 
\item 
\begin{verbatim}
// Java
class Sum {
    public static void main(String[] s) {
        int[] anArray = {4,55,19,23};
        int sum = 0;
        for (int item: anArray){
            sum += item;
        }
        System.out.println(sum);
    }
}

"back to Smalltalk: the version with inject:into:"
|anArray accum|
anArray := #(4 55 19 23).
anArray inject: 0 into: [:accum :elt | accum + elt]
^accum

"the same in the Workspace (without local variables)"
accum := 0.
accum := anArray inject: 0 into: [:accum :elt | accum + elt].
^accum.
\end{verbatim}
\end{enumerate}

\section*{Other exercises}
See last year.

\end{document}
\section*{Co/contra/nonvariance}
%http://archives.cs.iastate.edu/documents/disk0/00/00/02/53/00000253-00/tr01-10.pdf
%http://ece.colorado.edu/~siek/pubs/pubs/2005/siek05\_thesis.pdf
%http://c2.com/cgi/wiki?LanguageTypeErrors
%Java generics
\begin{enumerate}
\item The set of strings form a monoid under 
concatenation, with the empty string as identity. 
Other monoids (int,+,0), (int,*,1), (float, +,0), (list, append, emptyList),
\ldots 
\item Error: C\# requires invariance both for argument types and for
the return types. That's why we get an error for each method.

\item If C\# would allow covariance in the argument type,  one
could substitute siblings for another and that way cause a run-time
error. In the example below, the method
a.op() expects an IntegerAdditiveMonoid, but gets a RealAdditiveMonoid.
\begin{java}
struct RealAdditiveMonoid: Monoid  {
    public RealAdditiveMonoid(double x) {
         elt = x;
    }
    public RealAdditiveMonoid op(Monoid second) {
         return new RealAdditiveMonoid(
             elt + ((RealAdditiveMonoid)second).elt);
    }
    public RealAdditiveMonoid id(){
         return new RealAdditiveMonoid(0.0);
    }
    double elt;
};


class Main {
   public static void main(string[] args) {
      Monoid a = new IntegerAdditiveMonoid(7);
      Monoid b = new RealAdditiveMonoid(2.3);
      b = a.op(b);
   }
}

\end{java}
\item Java is covariant in the return type (JLS, 8,4,8.3). Therefore,
Java would only flag an error for the method op() (since it doesn't implement
the interface); the method id(), on the other hand, would be considered
an implementation. 
\item The typical workaround: use the requested signature, but then
cast the argument to the desired type (IntegerAdditiveMonoid),
and from there back to Monoid. Obviously, the first cast isn't safe. 
\begin{java}
    public Monoid op(Monoid second) { 
         return new IntegerAdditiveMonoid(
              elt + ((IntegerAdditiveMonoid)second).elt);
    }

    public Monoid id(){ 
         return new IntegerAdditiveMonoid(0);
    }
    int elt;
\end{java}
\end{enumerate}

\section*{Dispatch}
%http://www.disi.unige.it/person/ZuccaE/Didattica/Dottorato02/MultiJavaSlides.pdf
%http://www.itu.dk/courses/VOP/F2006/4_Slides.pdf
%http://blogs.msdn.com/devdev/archive/2005/08/29/457798.aspx
\begin{enumerate}
\item Problems of single dispatch: 1. Extensibility (if a new class is
derived from Shape, Rectangle and Circle have to change their implementation).
2. Run-time overhead due to type cast 3. Loss of static type safety
\begin{java}
class Shape {
    public boolean intersect(Shape s) {
        // intersect Shape x Shape
        return false;
    }
}
class Rectangle extends Shape {
    public  boolean intersect(Shape s) {
        if (s instanceof Rectangle) {
           // intersect Rectangle x Rectangle
        }
        else {
	    // intersect Rectangle x Shape
             return super.intersect(s);
        }
        return false;
    }
}
class Circle extends Shape {
   public boolean intersect(Shape s) {
         if (s instanceof Circle) {
           // intersect Circle x Circle
        }
        else {
             return super.intersect(s);
        }
	 return false;
    }
}
\end{java}
\item Problems with  double dispatch: 1.  Root class needs to be modified
each time a new class is added; 2. lots of code to write

C\# has different access rules: within a subclass, one can access 
protected members of the superclass only through  instances of the
subclass. Double-dispatching, however, wants to return control to 
the superclass, thus needs to take a superclass object. For C\#, 
it follows that the doubly-dispatched methods must be public. 
\begin{java}
class Shape {
    public  boolean intersect(Shape s) {
         return s.intersectShape(this);
    }
    protected  boolean intersectShape(Shape s) {
        // intersect Shape x Shape
        return true;
    }
    protected  boolean intersectRectangle(Rectangle r) {
        // no special code for Shape x Rectangle
        return intersectShape(r);
    }
    protected  boolean intersectCircle(Circle c) {
        return intersectShape(c);
   }
}
class Rectangle extends Shape {
    public  boolean intersect(Shape s) {
        return s.intersectRectangle(this);  
    }
    protected  boolean intersectRectangle(Rectangle r) {
        // specialized code for two rectangles
        return false; 
    }
}
class Circle extends Shape { 
    public  boolean intersect(Shape s) {
        return s.intersectCircle(this);  
    }
    protected  boolean intersectCircle(Circle c) {
        return false;
   }      
  
}
\end{java}
\item Problems with the visitor: 1. Needs to be anticipated at class
design time (of Shape). 2. Doesn't work so smoothly if one has to pass
additional parameters.
\begin{java}
class Shape {
    public void accept(IntersectVisitor v) {
	v.visitShape(this);
    }
}
class Rectangle extends Shape {
    public  void accept(IntersectVisitor v) { 
      v.visitRectangle(this);
    }
}
class Circle extends Shape {
    public  void accept(IntersectVisitor v) { 
      v.visitCircle(this); } 
}
 
// visitor hierarchy
abstract class IntersectVisitor { 
    boolean intersects; 
    public abstract void visitShape(Shape s);
    public abstract void visitRectangle(Rectangle r); 
    public abstract void visitCircle(Circle c); 
} 
class RectangleIntersectVisitor extends IntersectVisitor {
    public  void visitShape(Shape s) {
       // set intersects
   }
   public  void visitRectangle(Rectangle r) {
       // set intersects
   }
    public  void visitCircle(Circle r) {
       // set intersects
   }
}
class CircleIntersectVisitor extends IntersectVisitor { 
   public  void visitShape(Shape s) {
       // set intersects
   }
    public  void visitCircle(Circle r) {
       // set intersects
   }
   public  void visitRectangle(Rectangle r) {
       // set intersects
   }
}


class Intersect {
   public static void Main() {
      Shape s1 = new Rectangle(); 
      Shape s2 = new Circle();
      /* 
      boolean b = s2.intersect(s1);
      b = s1.intersect(s1);
      b = s2.intersect(s1);
      */
      IntersectVisitor v = new RectangleIntersectVisitor();
      s1.accept(v);
      if (v.intersects) {
          ; 
      }
   }
}
\end{java}

\end{enumerate}



\end{document}

