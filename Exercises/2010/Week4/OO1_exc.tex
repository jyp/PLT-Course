\documentclass{article}

\title{Programming Paradigms\\--\\OO Part I: Basic Imperative and OO Concepts}
\date{18th October 2010}

\input{header}


\begin{document}

\maketitle


\section{Activation records and scoping}
The \textit{scope} of a variable declaration refers to the range of
statements in which the variable can be referenced. 
\begin{itemize}
\item In \textit{static scoping}, the use of a variable $y$ in
a subprogram $f$ refers to the declaration of $y$ at the beginning of $f$ 
or, if not
declared there, to the declaration of $y$ in the subprogram $g$ 
whose declaration
contains the declaration of $f$, and so on. The static scope, thus,
is a property of the program text and can be determined at compile time.
\item In \textit{dynamic scoping}, on the other hand, 
the scope depends on the call sequence of
subroutines: the use of a variable refers to the declaration in 
the nearest caller (``dynamic parent'').
\end{itemize}
Java, C, \Cpp, C\#, and most other
 languages support static scoping. The language Perl is special
since it gives the user control over  scoping: 
variables declared as \textsf{`my'} are statically scoped, 
variables declared as \textsf{'local'} are dynamically scoped.  

\mycomment{
In this exercise, the question is: given a particular  use of
a variable, how does the run-time system know which declaration that 
variable refers to? As discussed in class, activation records contain
only two kinds of internal information: the return address and the previous
activation record. This does not suffice for languages with
static scoping and \textit{non-local variable references}.
}

\begin{enumerate}
\item What is the effect of static and dynamic scoping?
Consider the Perl program below and vary  between 
\textsf{my} and \textsf{local}
for the three
declarations of $x$. When, if at all, does the value $x=2$ get printed?
When $x=4.14$ or $x=56$ (if at all)? 
(Hint: Perl is installed on (almost) all machines; you can run
the program.)
\item In dynamic scoping, it is easy for the run-time system
to determine which declaration
the use of a variable refers to. Explain. (Hint: run-time
stack)
\item Static scoping with non-local references, on the other hand, is
  not so easy to implement. In particular, the kind of activation
  records discussed in class (which contain only two kinds of internal
  information: the return address and the previous activation record)
  does not suffice.
%

What information would be necessary and where could it be stored? 
Make a suggestion  and 
sketch a (simple) algorithm that takes a run-time stack and an activation record,
and returns the ``right'' reference (or at least its activation record).
How expensive is your algorithm? Determine the parameters on which its
cost depends and give the cost of the algorithm.
\end{enumerate}


\begin{perl}
 sub outer {
   my $x = 0;

   sub r(){
       $x = $x + 1; 
       print("In subroutine r, x = ", $x);
   }
   sub q() {
      my $x = 3.14; 
      &r;      # call outer's subroutine r
   }
   sub p() {
      my $x = 55; 
      &q;      # call outer's subroutine q
   } 
   &p;         # call subroutine p
}

&outer;
\end{perl}


\section{Parameter passing}
\begin{enumerate}
\item For each mechanism of parameter passing, determine the final
values of the array \textsf{A} after the call to \textsf{f}. 

\begin{java}
void f(int x, int y) {
    x = 1;
    y = 3;
}

int[] A = new int[2];
A[0] = 0;
A[1] = 2;
f(A[0],A[A[0]]);
\end{java}
\item An important question of call-by-result, which was not discussed in
class, is the question \textit{when} the lvalue of the actual parameter is 
computed. One might expect that this happens before the callee executes,
but that is not necessarily the case---since the value of the actual
parameter is not needed until the 
callee returns, its computation could be delayed until the value is needed.
Ada, for example,  supports early lvalue-computation, while Algol-W implements
late lvalue-computation.

Use the snippet below to explain the difference between early and late
lvalue-computation.
\begin{java}
void f(int [] A) {
   int i = 0;
   g(i, A[i]);
}
void g(by-reference int i, by-result int j) {
   j = 2;
   i = 1;
   j = 3;
}
\end{java}

\item Another important detail of call-by-result and call-by-value-result concerns
the order in which the final values of the formal parameter are written back
to the actual parameters. For a binary method \textsf{f(x,y)}, for
example, either (the value of) \textsf{x} or (the value of) 
\textsf{y} is written back
first; this order might make a difference. 

Write two methods \textsf{f,g} (in pseudo-code,
if you wish) that demonstrate the difference.
\end{enumerate}

\section{An Introduction to Squeak}
\begin{enumerate}
\item For each of the following expressions determine
the message(s), and for each message determine the receiver object and all 
argument object(s). 
\begin{java}
Date today 
2 + 3 
anArray at: 1 put: 'hej'
10 raisedTo: 2 + 3
10 negated raisedTo: 2 + 3
2 * 3 + 4      
\end{java}
\item The following snippet contains a fragment of the class \textsf{Count} with the methods \textsf{initialize},
 \textsf{value},  \textsf{increment1}, and \textsf{increment2}:
\begin{java}
initialize
    value := 0.
value
    value notNil ifFalse: [^0].
    ^value
increment1
    value := value + 1.
increment2
    value := self value + 1
\end{java}
\begin{enumerate}
\item Compare the two increment
methods from a design point of view: which one would you prefer?  Why?
\item What is the meaning of the semicolon (``;''), thus  the value of \textsf{aCounter} at the end of  the following program (1 sentence)?

\begin{java}
|aCounter|
aCounter := Count new.
aCounter initialize.
aCounter increment1; increment2.
\end{java}
(Hint: open the ``Browser'', implement the class \textsf{Count}, enter the
 program below in 
the ``Workspace'',
run the program, and ``inspect'' the counter variable.)
\end{enumerate}
\item Consider the following block expression
\begin{java}
|anArray sum|
sum := 0
anArray := #(4 55 19 23).
anArray do: [:item | sum := sum + item].
^sum
\end{java}
\begin{enumerate}
\item Rewrite the block in Java (or a Java-like language). (This is just
a sanity check; don't make it too complicated!)
\item Rewrite the block using \textsf{inject:into}.
%\item Rewrite the block using explicit array indexing (via the method \textsf{at:}) to access array elements. 
\end{enumerate}
\item Implementing an abstract syntax tree for a small calculator
  language. The language grammar looks like
  $$\begin{array}{lcl}
    e & ::= & e + e | e - e | e * e | e / e \\
  \end{array}$$
  \begin{itemize}
    \item Implement a class \textsf{ASTNode} with methods for getting
      the number of children \textsf{childrenCount}, setting a child
      \textsf{setChild:At:} and retrieving a child \textsf{childAt:}
   \item Implement for each operator $+,-$ etc.\ an own subclass of
      ASTNode \textsf{AddNode},\textsf{MinusNode} etc. as well as one
      for integers \textsf{IntLitNode}. 
    \item Implement an interpreter for the language by implementing
      the method \textsf{eval} for each of the classes which evaluates
      first the children (by subsequent calls of \textsf{eval}) and,
      finally, computes and returns the result.
    \item Implement a short test class which constructs the ASTs for 
      \begin{enumerate}
        \item $1 + (2 * 3)$
        \item $(1 + 2) * 3$
        \item $(2 * 119) / 17$
     \end{enumerate}
     evaluates them and prints the result of the evaluation.
  \end{itemize}
\end{enumerate}
Further material
\begin{itemize}
\item By default, Squeak is not installed on the machines in the department,
but you can easily install it yourself at home or in your account at
Chalmers. Visit the download area on the main webpage of Squeak:
\begin{center}
\textsf{http://www.squeak.org}
\end{center}
\item The Squeak main page contains links to tutorials of varying degree
of difficulty and lots of other useful online resources. Here is a 
tutorial for Java programmers:
\begin{center}
\textsf{http://daitanmarks.sourceforge.net/or/squeak/squeak\_tutorial.html}
\end{center}
\end{itemize}

\end{document}


\section{Answers (sketches)}


\section*{Scoping}
\begin{enumerate}
\item Scoping declaration in outer/p/q
\begin{itemize}
\item my/my/my  or local/my/my -$>$ x=1
\item local/local/local or local/my/local -$>$ x =4.14
\item local/local/my -$>$ x=55
\end{itemize}
\item Traverse the run-time stack, take the first declaration encountered.
\item (Not-constant solution:) Include in the activation record a 
``static link'' (a.k.a. access links), then follow the link to the
static parent. For a solution in constant time: compute the
``static distance'' between a variable definition and a variable
reference; jump to the definition in constant time.
\end{enumerate}


\section*{Parameter passing}
\begin{enumerate}
\item Elements in A

\begin{tabular}{lll}
               & A[0] & A[1] \\
 by-reference  & 3    & 2    \\
by-value        & 0   & 2   \\
by-result      &  3 (1) & 2 (3);  depends on order, see next problem \\
by-value-result & 1 & 3 \\
by-name      & 1 & 3  \\
\end{tabular}
\item Early/late:

\begin{tabular}{lll}
               & A[0] & A[1] \\
early          & 3    & ? \\
late       & ?   & 3   \\
middle     &    & \\
\end{tabular}
\item Order:
\begin{java}
int f(int x, int y) {
    x = 4;
    y = 5;
}
int a=3;
p(a,a); 
\end{java}

\end{enumerate}


\end{document}
